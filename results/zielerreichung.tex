\section{Zielerreichung}

Im Rahmen dieser Arbeit wurde eine Android Applikation entwickelt die dem neusten Stand der Technik entspricht und auf allen gängigen Android Smartphones der letzten 2 Jahre lauffähig ist. \footnote{Android 4.3 und höher wird unterstützt.} Alle beim Kick-Off Meeting ausgearbeiteten Anforderungen des Kunden (siehe \ref{sec:Functional Requirements}) wurden erfolgreich umgesetzt und getestet.\footnote{Anforderungen der Priorität 3 sind konzeptionell unterstützt.}

Kurz zusammengefasst hat die Applikation die folgenden Features:

\begin{itemize}
\item Ein Service Techniker kann eine neue Küche erfassen und diese mit Beschreibung und Bild versehen. In der erfassten Küche können dann Fotos der verschiedenen Küchenbereiche gespeichert werden. 
\item Durch ein Bluetooth Discovery Vorgang werden alle vorhandenen, aktiven Fluxron Küchengeräte erkannt. Diese können frei auf den Fotos der Küchenbereiche platziert werden.
\item Nachdem Geräte erfasst wurden ist auf den Fotos der Küchenbereichen jederzeit der Betriebsstatus ersichtlich. 
\item Detail Informationen wie Temperatur, Betriebsmodus, Fehlermeldung und ähnliches werden sichtbar wenn man auf das Icon eines Geräts drückt. Auch Änderungen an der Konfiguration des Geräts sind möglich.
\item Bereits erfasste Küchen können zwischen den Service Technikern einer Firma durch eine Export/Import Funktion ausgetauscht werden.
\item Natürlich sind auch alle denkbaren \ac{CRUD}-Operationen unterstützt.
\end{itemize}

Da eine Weiterentwicklung durch die Fluxron vorgesehen ist, wurde bei der Programmierung speziellen Wert darauf gelegt, dass die Applikation gut dokumentiert und leicht erweiterbar ist. Der JavaDoc Abdeckungsgrad liegt bei nahezu 100\%.