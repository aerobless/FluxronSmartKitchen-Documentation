\section{Arbeitspakete}
\label{sec:Arbeitspakete}

Entsprechend zu den Meilensteinen wurden die folgenden Arbeitspakete geplant:

\subsection{Analyse}
\begin{table}[H]
\begin{tabularx}{\textwidth}{ c | l | X }

\textbf{Nr} & \textbf{Name} & \textbf{Inhalt} \\ \hline
01 & Ausgangslage & Kick-Off und Einarbeitung \\ \hline
02 & Infrastruktur aufsetzen & VPS in Betrieb nehmen, Redmine installieren, Code und Dokumentation Repositories konfigurieren, Backup-Strategie erstellen. \\ \hline
03 & Arbeitspakete & Issues in Redmine erfassen und Arbeitspakete dokumentieren. \\ \hline
04 & Use Cases & Use Cases und User Stories anhand der Aufgabenstellung entwerfen und dokumentieren.\\ \hline
05 & Anforderungsspezifikation & \ac{FR} und \ac{NFR} dokumentieren.\\ \hline
06 & Testspezifikation & Aufgrund der \ac{FR} und \ac{NFR} eine Testspezifikation entwerfen.\\ \hline
07 & Projektplan & Ein Projektplan erstellen die Arbeitspakete in einen zeitlichen Kontext bringt.\\ \hline
08 & Risikomanagement & Mögliche Risiken in Erfahrung bringen und dokumentieren.\\ \hline
09 & Domainanalyse & Ein Domain-Modell der zu modellierenden Umgebung erstellen.\\ \hline
10 & Meilensteine & Meilensteine erfassen und dokumentieren.\\
\end{tabularx}
\caption{Arbeitspaket Analyse}
\end{table}