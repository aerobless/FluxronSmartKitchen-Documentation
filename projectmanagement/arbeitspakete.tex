\section{Arbeitspakete}
\label{sec:Arbeitspakete}

Entsprechend zu den Meilensteinen wurden die folgenden Arbeitspakete geplant:

\subsection{Analyse}
\begin{table}[H]
\begin{tabularx}{\textwidth}{ c | l | X }
\textbf{Nr} & \textbf{Name} & \textbf{Inhalt} \\ \hline
01 & Ausgangslage & Kick-Off und Einarbeitung \\ \hline
02 & Infrastruktur aufsetzen & VPS in Betrieb nehmen, Redmine installieren, Code und Dokumentation Repositories konfigurieren, Backup-Strategie erstellen. \\ \hline
03 & Use Cases & Use Cases und User Stories anhand der Aufgabenstellung entwerfen und dokumentieren.\\ \hline
04 & Anforderungsspezifikation & \ac{FR} und \ac{NFR} dokumentieren.\\ \hline
05 & Testspezifikation & Aufgrund der \ac{FR} und \ac{NFR} eine Testspezifikation entwerfen.\\ \hline
06 & Domainanalyse & Ein Domain-Modell der zu modellierenden Umgebung erstellen.\\ \hline
07 & Meilensteine & Meilensteine erfassen und dokumentieren.\\
\end{tabularx}
\caption{Arbeitspaket Analyse}
\end{table}

\subsection{Projektmanagement}
\begin{table}[H]
\begin{tabularx}{\textwidth}{ c | l | X }
\textbf{Nr} & \textbf{Name} & \textbf{Inhalt} \\ \hline
10 & Arbeitspakete & Issues in Redmine erfassen und Arbeitspakete dokumentieren. \\ \hline
11 & Projektplan & Ein Projektplan erstellen die Arbeitspakete in einen zeitlichen Kontext bringt.\\ \hline
12 & Risikomanagement & Mögliche Risiken in Erfahrung bringen und dokumentieren.\\
\end{tabularx}
\caption{Arbeitspaket Projektmanagement}
\end{table}

\subsection{Design}
\begin{table}[H]
\begin{tabularx}{\textwidth}{ c | l | X }
\textbf{Nr} & \textbf{Name} & \textbf{Inhalt} \\ \hline
20 & Guidelines definieren & Coding und Style Guidelines definieren.\\ \hline
21 & Architektur & Applikationsarchitektur definieren und begründen. Designentscheide und Patterns festhalten.\\ \hline
22 & Evaluation Libraries & Libraries zur Unterstützung der Architektur evaluieren.\\ \hline
23 & Nebenläufigkeitskonzept & Konzept zur Handhabung von Nebenläufigkeiten ausarbeiten.\\ \hline
24 & UI Designstudie & Konzept für das Benutzerinterface entwerfen.\\ \hline
25 & Architekturprototyp & Implementation des Architekturprototypen.\\ 
\end{tabularx}
\caption{Arbeitspaket Design}
\end{table}

\subsection{Implementation}
\begin{table}[H]
\begin{tabularx}{\textwidth}{ c | l | X }
\textbf{Nr} & \textbf{Name} & \textbf{Inhalt} \\ \hline
30 & Implementation 1 & Erste Implementations-Iteration. \\ \hline
31 & Implementation 2 & Zweite Implementations-Iteration.\\ \hline
32 & Implementation 3 & Dritte Implementations-Iteration. \\ \hline
33 & Implementation 3 & Dritte Implementations-Iteration. 75\% der Features sind implementiert. \\ \hline
34 & Implementation 4 & Vierte Implementations-Iteration. Alle Features sind implementiert. \\ 
\end{tabularx}
\caption{Arbeitspaket Implementation}
\end{table}

\subsection{Dokumentation}
\begin{table}[H]
\begin{tabularx}{\textwidth}{ c | l | X }
\textbf{Nr} & \textbf{Name} & \textbf{Inhalt} \\ \hline
40 & Zwischenpräsentation & Präsentation vorbereiten und halten.\\ \hline
41 & Benutzerhandbuch & Benutzerhandbuch zur Applikation erstellt.\\ \hline
42 & Abstract & Abstract geschrieben und bereit zur Abgabe.\\ \hline
43 & Bericht & Dokumentation geschrieben und bereit zur Abgabe.\\ \hline
44 & Schlusspräsentation & Präsentation vorbereiten und halten.\\ 
\end{tabularx}
\caption{Arbeitspaket Dokumentation}
\end{table}