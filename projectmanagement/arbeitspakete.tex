\section{Arbeitspakete}
\label{sec:Arbeitspakete}

Entsprechend zu den Meilensteinen wurden die folgenden Arbeitspakete geplant:

\subsection{Analyse}
\begin{table}[H]
\begin{tabularx}{\textwidth}{ c | l | X }
\textbf{Nr} & \textbf{Name} & \textbf{Inhalt} \\ \hline
01 & Ausgangslage & Kick-Off und Einarbeitung \\ \hline
02 & Infrastruktur aufsetzen & \acs{VPS} in Betrieb nehmen, Redmine installieren, Code und Dokumentation Repositories konfigurieren, Backup-Strategie erstellen. \\ \hline
03 & Use Cases & Use Cases und User Stories anhand der Aufgabenstellung entwerfen und dokumentieren.\\ \hline
04 & Anforderungsspezifikation & \ac{FR} und \ac{NFR} dokumentieren.\\ \hline
05 & Testspezifikation & Aufgrund der \ac{FR} und \ac{NFR} eine Testspezifikation entwerfen.\\ \hline
06 & Domainanalyse & Ein Domain-Modell der zu modellierenden Umgebung erstellen.\\ \hline
07 & Meilensteine & Meilensteine erfassen und dokumentieren.\\
\end{tabularx}
\caption{Arbeitspaket Analyse}
\end{table}

\subsection{Projektmanagement}
\begin{table}[H]
\begin{tabularx}{\textwidth}{ c | l | X }
\textbf{Nr} & \textbf{Name} & \textbf{Inhalt} \\ \hline
10 & Arbeitspakete & Issues in Redmine erfassen und Arbeitspakete dokumentieren. \\ \hline
11 & Projektplan & Projektplan erstellen, der die Arbeitspakete in einen zeitlichen Kontext bringt.\\ \hline
12 & Risikomanagement & Risiken in Erfahrung bringen und dokumentieren.\\
\end{tabularx}
\caption{Arbeitspaket Projektmanagement}
\end{table}

\subsection{Design}
\begin{table}[H]
\begin{tabularx}{\textwidth}{ c | l | X }
\textbf{Nr} & \textbf{Name} & \textbf{Inhalt} \\ \hline
20 & Guidelines definieren & Coding und Style Guidelines definieren.\\ \hline
21 & Architektur & Applikationsarchitektur definieren und begründen. Designentscheide und Patterns festhalten.\\ \hline
22 & Evaluation Libraries & Libraries zur Unterstützung der Architektur evaluieren.\\ \hline
23 & Nebenläufigkeitskonzept & Konzept zur Handhabung von Nebenläufigkeiten ausarbeiten.\\ \hline
24 & UI Designstudie & Konzept für das Benutzerinterface entwerfen.\\ \hline
25 & Architekturprototyp & Applikationsstruktur und Layers aufbauen. Eventbus einrichten. Via Bluetooth Werte auf dem Testgerät abfragen.\\ 
\end{tabularx}
\caption{Arbeitspaket Design}
\end{table}

\pagebreak
\subsection{Implementation}

Beim implementieren der Applikation sind wir agil vorgegangen und haben mit Iterationen von jeweils einer Woche gearbeitet. Der Inhalt der einzelnen Iterationen wurde daher eine Woche vor dem Beginn der Phase konkretisiert und hier entsprechend nachgeführt.

\begin{table}[H]
\begin{tabularx}{\textwidth}{ c | l | X }
\textbf{Nr} & \textbf{Name} & \textbf{Inhalt} \\ \hline
30 & Implementation 0 & Grundlayout für alle Activities. Layout Küchenliste. EDS Files einlesen und parsen. Fotofunktion für \enquote{Küche erstellen}. DeviceManager verwaltet Geräte.\\ \hline
31 & Implementation 1 & Grundlayout Küchenübersicht. Ankommende Nachrichten interpretieren. Checksummen-Prüfung von Nachrichten. Thumbnails von Bildern laden. Codegenerator für Parameter.\\ \hline
32 & Implementation 2 & Geräte platzieren und skalieren. Bluetooth Verbindungen cachen. Küchenbereich hinzufügen und speichern. Device Discovery UI. Geräte validieren.\\ \hline
33 & Implementation 3 & 75\% der Features sind implementiert. Device Activity mit Tabs. Konsistentes Navigationsdesign. Werte auf Gerät ändern können. Automatisiertes Pairing. Parameter zyklisch aktualisieren. \\ \hline
34 & Implementation 4 & Parameter in Device Ansicht darstellen. UI animieren. Layouts für Device Usage, Errors und Status. Custom Control zur Darstellung der Temperatur.\\ \hline
35 & Implementation 5 & Alle Features sind implementiert. Klassenerkennung bei Geräten. Schaltflächen stylen. Fixtexte konsequent im Ressourcen File hinterlegen. Error-Handling für Gerätefehler. Prozentzahlen für Nutzungsstatistik berechnen.\\
\end{tabularx}
\caption{Arbeitspaket Implementation}
\end{table}

\subsection{Testing}

\begin{table}[H]
\begin{tabularx}{\textwidth}{ c | l | X }
\textbf{Nr} & \textbf{Name} & \textbf{Inhalt} \\ \hline
40 & Bugfixes 0 & Applikation mit lokalem Testaufbau testen. Kritische Bugs beheben. Sicherstellen, dass alle Muss-Features implementiert wurden.\\ \hline
41 & Testtag bei Fluxron & Applikation mit echten Kochherden testen. Vorgehen gemäss der \nameref{s:Systemtest_Spezifikation}. \\ \hline
42 & Bugfixes 1 & Restliche Bugs beheben. Design vereinheitlichen. Sicherstellen, dass der ganze Code mit JavaDocs versehen ist.\\ \hline
\end{tabularx}
\caption{Arbeitspaket Testing}
\end{table}

\subsection{Dokumentation}
\begin{table}[H]
\begin{tabularx}{\textwidth}{ c | l | X }
\textbf{Nr} & \textbf{Name} & \textbf{Inhalt} \\ \hline
50 & Zwischenpräsentation & Präsentation vorbereiten und halten.\\ \hline
51 & Benutzerhandbuch & Benutzerhandbuch zur Applikation erstellt.\\ \hline
52 & Abstract & Abstract geschrieben und bereit zur Abgabe.\\ \hline
53 & Bericht & Dokumentation geschrieben und bereit zur Abgabe.\\ \hline
54 & Schlusspräsentation & Präsentation vorbereiten und halten.\\ 
\end{tabularx}
\caption{Arbeitspaket Dokumentation}
\end{table}