% Risikomanagement

\section{Risikomanagement}
\label{sec:Risikomanagement}

\begin{table}[H]
\begin{tabularx}{\textwidth}{l|>{\raggedright\arraybackslash}X}
\multicolumn{2}{l}{\textbf{R1: Erwartungen des Kunden nicht erfüllt }} \\
\hline
Beschreibung & Die Applikation erfüllt die funktionalen oder gestalterischen Erwartungen des Kunden nicht.\\
\hline
Massnahme & Es wird mit dem Kunden eine wöchentliche Besprechungen per Skype durchgeführt. Dabei wird der aktuelle Stand der Arbeit gezeigt.\\
\hline
Vorgehen beim Eintreffen & Applikation gemäss den Wünschen des Kunden anpassen.
\\
\end{tabularx}
\caption{Risiko - Erwartungen des Kunden nicht erfüllt}
\end{table}

\begin{table}[H]
\begin{tabularx}{\textwidth}{l|>{\raggedright\arraybackslash}X}
\multicolumn{2}{l}{\textbf{R2: Performance reicht nicht für das Verwalten von 100 Geräten }} \\
\hline
Beschreibung & Die Performance der Applikation reicht nicht aus um 100 Fluxron Geräte pro Küche zu verwalten beziehungsweise grafisch darzustellen.\\
\hline
Massnahme & Bei der Entwicklung werden Performance Tests mit bis zu 100 simulierten Fluxron Geräten durchgeführt.\\
\hline
Vorgehen beim Eintreffen & Die maximale Zahl von Geräten pro Küche reduzieren.
\\
\end{tabularx}
\caption{Risiko - Performance reicht nicht}
\end{table}

\begin{table}[H]
\begin{tabularx}{\textwidth}{l|>{\raggedright\arraybackslash}X}
\multicolumn{2}{l}{\textbf{R3: Bluetooth Classic und 4.0 nicht in der gleichen Applikation }} \\
\hline
Beschreibung & Es ist nicht möglich Unterstützung für Bluetooth Classic und 4.0 in der gleichen Applikation anzubieten.\\
\hline
Massnahme & In der Elaboration Phase wird die Bluetooth Unterstützung von Android recherchiert. Es wird nach Möglichkeit eine Android Version gewählt die beide Bluetooth Standards unterstützt.\\
\hline
Vorgehen beim Eintreffen & Es wird nur Unterstützung für Bluetooth Classic angeboten.
\\
\end{tabularx}
\caption{Risiko - Bluetooth Classic \& 4.0 Unterstützung}
\end{table}