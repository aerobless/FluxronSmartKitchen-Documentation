% Index of /design/

\section{Mockups}
\label{sec:Mockups}

\subsection{Startbildschirm}
\label{subsec:Startbildschirm}

\begin{wrapfigure}[8]{r}{6cm}
	\begin{center}
		\includegraphics[page=1,trim=0 0 0 0,clip,scale=0.21]{uiux/res/mockups}
		\caption{Mockup Startbildschirm}
	    \label{abb:mockStartScreen}
	\end{center}
\end{wrapfigure}

Der Startbildschirm enthält ein Suchfeld zur einfachen Suche einer Küche nach dem Namen. Darunter werden die Küchen mit ihrem Namen, der Beschreibung und mit der Anzahl Geräte aufgelistet. 

Der Benutzer kann nun eine Küche durch Antippen öffnen, oder mittels dem \ac{FAB} eine neue Küche hinzufügen.

\WFclear
\vspace{3cm}

\subsection{Küchenerstellung}
\label{subsec:Küchenerstellung}

\begin{wrapfigure}[14]{r}{6cm}
	\begin{center}
		\includegraphics[page=2,trim=0 0 0 0,clip,scale=0.21]{uiux/res/mockups}
		\caption{Mockup Küchenerstellung}
	    \label{abb:mockCreateKitchen}
	\end{center}
\end{wrapfigure}

Bei der Küchenerstellung gibt der Benutzer einen Namen für die Küche an. Zudem kann der Benutzer, wenn er möchte, noch eine Beschreibung für diese Küche eingeben.

Zusätzlich sollte der Benutzer auch noch ein Foto von der Küche machen. Damit kann er die Küche später einfacher wiedererkennen. Küchenname und Bild müssen zwingend erfasst werden, die Beschreibung ist nicht erforderlich.

Durch Antippen des \enquote{Create}-Buttons wird die Küche erfasst und der Benutzer auf die Küchenansicht umgeleitet.

\WFclear

\subsection{Küchenbereich mit aktivem Suchlauf}
\label{subsec:Küchenbereich mit aktivem Suchlauf}

\begin{wrapfigure}[12]{r}{6cm}
	\includegraphics[page=4,trim=0 0 0 0,clip,scale=0.21]{uiux/res/mockups}
	\caption{Mockup Küchenansicht}
    \label{abb:mockKitchenView}
\end{wrapfigure}

In der Küchenansicht sieht der Benutzer das Foto für den aktuell gewählten Küchenbereich. Darauf kann er die Geräte aus der Geräteliste platzieren.Die Geräte werden durch einen Kreis, welcher den Status des Gerätes anzeigt, symbolisiert.

Durch Antippen eines Gerätes kann dieses nach Rückfrage gelöscht werden.

Die Geräteliste enthält alle Geräte, welche beim Suchlauf gefunden wurden, oder bereits in der Küche erfasst wurden.

\WFclear

\subsection{Gerätestatus}
\label{subsec:Gerätestatus}

\begin{wrapfigure}{r}{6cm}
	\begin{center}
		\includegraphics[page=5,trim=0 0 0 0,clip,scale=0.21]{uiux/res/mockups}
		\caption{Mockup Gerätestatus}
		\label{abb:mockDeviceDetail}
	\end{center}
\end{wrapfigure}

Die Statusansicht für ein Gerät besteht aus Tabs, zwischen denen der Benutzer mittels Swiping wechseln kann. Für jeden Gerätetyp braucht es hier eine eigene Ansicht, es sind für jede Art von Gerät andere Parameter und Werte relevant.

\begin{itemize}
  \item Statusübersicht mit Sensorwerten
  \item Nutzungsstatistik des Geräts
  \item Fehlerhistorie mit letzten 10 Fehlercodes
  \item Einstellungsseite mit Konfigurationsparametern
\end{itemize}

\vspace{6cm}
\WFclear

\subsection{Geräteparameter}
\label{subsec:Geräteparameter}

\begin{wrapfigure}{r}{6cm}
	\begin{center}
		\includegraphics[page=6,trim=0 0 0 0,clip,scale=0.21]{uiux/res/mockups}
		\caption{Mockup Geräteparameter}
		\label{abb:mockDeviceParams}
	\end{center}
\end{wrapfigure}

Die Parameteransicht zeigt eine Liste aller veränderbaren Konfigurationseinstellungen an. Der Benutzer kann die Parameter ändern oder zurücksetzen.

Falls ein Parameter nicht geschrieben werden kann wird eine Fehlermeldung ausgegeben.

Diese Ansicht ist ebenfalls für jeden Gerätetypen unterschiedlich, so haben z.B. Thermostaten mehrere Konfigurationsprofile, die alle in dieser Ansicht bearbeitet werden können.

\WFclear
\vspace{2cm}

\subsection{Nutzungsstatistik}
\label{subsec:Nutzungsstatistik}

\begin{wrapfigure}{r}{6cm}
	\begin{center}
		\includegraphics[page=7,trim=0 0 0 0,clip,scale=0.21]{uiux/res/mockups}
		\caption{Mockup Nutzungsstatistik}
		\label{abb:mockDeviceUsage}
	\end{center}
\end{wrapfigure}

Über die Nutzungsstatistik kann der Benutzer auslesen, wie lange die Werte der Sensoren in den einzelnen Temperaturbereichen lagen. Dies kann zum Beispiel Auskunft über eine fehlerhafte Installation (nicht genügend Wärmeabfuhr am Kühlkörper) Aufschluss geben.

Auch die Statistikseite kann von Gerätetyp zu Gerätetyp variieren.

\WFclear
\vspace{2cm}

\subsection{Fehlerhistorie}
\label{subsec:Fehlerhistorie}

\begin{wrapfigure}{r}{6cm}
	\begin{center}
		\includegraphics[page=8,trim=0 0 0 0,clip,scale=0.21]{uiux/res/mockups}
		\caption{Mockup Fehlerhistorie}
		\label{abb:mockDeviceErrors}
	\end{center}
\end{wrapfigure}

In der Fehlerhistorie werden dem Benutzer die letzten zehn Fehlercodes des Gerätes angezeigt. Diese Bestehen aus einem Fehlercode und einem beschreibenden Text. Zudem wird der Laufzeitzähler des Gerätes zum Fehlerzeitpunkt ausgegeben. Das heisst, für jeden Fehler kann nachgesehen werden, in welcher Laufzeitstunde (ab Installationszeitpunkt) der Fehler aufgetreten ist.

Einzelne Gerätetypen wie z.B. Thermostaten haben weniger oder mehr Fehlercodes in der Historie.