% Der Abstract richtet sich an den Spezialisten auf dem entsprechenden Gebiet
% und beschreibt daher in erster Linie die (neuen, eigenen) Ergebnisse und
% Resultate der Arbeit. Es umfasst nie mehr als eine Seite, typisch sogar nur
% etwa 200 Worte (etwa 20 Zeilen). Es sind keine Bilder zu verwenden.

\chapter*{Abstract}\addcontentsline{toc}{chapter}{Abstract}

Die Firma Fluxron Solutions AG entwickelt in Amriswil Heizlösungen und Küchengeräte auf Induktionsbasis. Diese Geräte besitzen eine Bluetooth-Schnittstelle, über welche ihre Einstellungen angepasst werden können. Zudem bietet die Schnittstelle ausführliche Laufzeit- und Fehlerprotokolle. Servicetechniker benötigen genau diese Informationen zur Fehlersuche und Reparatur der Geräte in Grossküchen. Aufgrund der grossen Geräteanzahl, ist es schwierig die Installationen im Überblick zu behalten.

In dieser Arbeit wurde eine Applikation für Android entwickelt, welche Techniker zur Diagnose und Konfiguration nutzen. Die Lage der eingebauten Geräte wird auf Situationsfotos markiert. Bei einem späteren Serviceeinsatz werden diese Positionen und der Status aller Kochinstallationen abgerufen.

Zur Umsetzung des Projektes wurden agile Softwareentwicklungsmethoden eingesetzt. Neben einer gründlichen Anforderungsanalyse wurde die Benutzeroberfläche mit Mockups konzipiert und mittels Usability-Walkthrough validiert.

Als Programmiersprache wurde Java 7 für Android eingesetzt. Die Anwendungsarchitektur besteht aus drei Layern, welche mittels Messages über ein Event Bus System kommunizieren. Zur Erstellung einer zeitgemässen Benutzeroberfläche, wurde diese im effektiven Material Design umgesetzt. Lokal werden die Küchendaten in einer dokumentbasierten Datenbank gespeichert. Die Kommunikation mit den Geräten erfolgt über das CANopen-Protokoll. Zudem bietet die Anwendung konzeptionelle Unterstützung für die Anbindung an ein Cloud-Backend.

Der Funktionsumfang der Mobilapplikation umfasst die Verwaltung mehrerer Küchen und der darin installierten Geräte. Küchen werden zur besseren Übersicht in einzelne Bereiche unterteilt. Geräte eines Bereichs werden regelmässig zur Statusaktualisierung abgefragt. Der Funktionsumfang wurde mit einem erfolgreichen Praxistest vor Ort überprüft. \todo{V1 - AKTUALISIERUNG NÖTIG!}