% Der Abstract richtet sich an den Spezialisten auf dem entsprechenden Gebiet
% und beschreibt daher in erster Linie die (neuen, eigenen) Ergebnisse und
% Resultate der Arbeit. Es umfasst nie mehr als eine Seite, typisch sogar nur
% etwa 200 Worte (etwa 20 Zeilen). Es sind keine Bilder zu verwenden.

\chapter*{Abstract}\addcontentsline{toc}{chapter}{Abstract}
Die Firma Fluxron Solutions AG entwickelt in Amriswil Heizlösungen und Küchengeräte auf Induktionsbasis. Diese Geräte besitzen eine Bluetooth-Schnittstelle, über welche Einstellungen angepasst und ausführliche Laufzeit- und Fehlerprotokolle ausgelesen werden können. Servicetechniker benötigen genau diese Informationen zur Reparatur der Geräte in Grossküchen. Aufgrund der grossen Geräteanzahl, ist es schwierig die Installationen im Überblick zu behalten.

In dieser Arbeit wurde eine Applikation für Android entwickelt, welche Techniker zur Diagnose und Konfiguration nutzen. Die Lage der eingebauten Geräte wird auf Situationsfotos markiert. Bei einem späteren Serviceeinsatz werden diese Positionen und der Status aller Kochinstallationen abgerufen.

Zur Umsetzung des Projektes wurden agile Softwareentwicklungsmethoden eingesetzt. Neben einer gründlichen Anforderungsanalyse wurde die Benutzeroberfläche mit Mockups im Material Design konzipiert und mittels Usability-Walkthrough validiert.

Als Programmiersprache wurde Java 7 für Android eingesetzt. Die Anwendungsarchitektur besteht aus drei Layern, welche mittels Messages über ein Event Bus System kommunizieren. Lokal werden die Küchendaten in einer dokumentbasierten Datenbank gespeichert. Die Kommunikation mit den Geräten erfolgt über das CANopen-Protokoll. Zudem wurde die Architektur darauf ausgelegt, die Erweiterung um ein Cloud-Backend einfach zu machen.

Der Funktionsumfang der Mobilapplikation umfasst die Verwaltung mehrerer Küchen und der darin installierten Geräte. Küchen werden in einzelne Bereiche unterteilt und der Status der Geräte wird regelmässig aktualisiert. Der Funktionsumfang wurde mit einem erfolgreichen Praxistest vor Ort überprüft. Die Servicetechniker profitieren nun von einer modernen Applikation, welche ihnen den Wartungsalltag erleichtert.