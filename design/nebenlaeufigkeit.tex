\section{Nebenläufigkeitskonzept}
\label{sec:Nebenläufigkeitskonzept}
Das Nebenläufigkeitskonzept bezieht sich auf Architektur Variante C \enquote{Event Bus}.

\subsection{Sicherstellen der Asynchronität}
Um sicherzustellen das alle Ereignisse asynchron verarbeitet werden, muss der Event Bus so konfiguriert werden dass er die Meldungsverarbeitung in einem eigenen Thread durchführt. Dies gilt auch für Meldungen die im \ac{UI}-Thread erzeugt werden. Damit ist sichergestellt das der \ac{UI}-Thread nicht blockiert.

\subsection{Synchronisierung mit \ac{UI}-Thread}
Zugriffe auf das User Interface dürfen nur vom UI-Thread durchgeführt werden. Dazu müssen die Subscriber dem Event Bus den gewünschten Ausführungsthread angeben können.

\subsection{Thread-Safety der Event Bus Subscriber}
Alle Subscribers müssen Sicherstellen (mit Ausnahme UI-Komponenten) dass von mehreren Threads aufgerufen werden können. Dazu werden klassische Java Monitors verwendet. Die Subscribers folgen dem "Run-to-completion" Prinzip.

\subsection{Restart von Activities}
Android kann zur Speicheroptimierung laufende Applikationen (im Hintergrund) ganz oder teilweise beenden. Öffnet der Benutzer die Applikation erneut muss sie ihren Zustand wiederherstellen. Dies erfordert eine Überprüfung dass die Instanzen im Application Context noch vorhanden sind. Gegebenenfalls müssen die Event Busse neu gestartet werden.