\section{Code Style Guidelines}
\label{sec:Code Style Guidelines}

Um die Zusammenarbeit mit mehreren Personen zu regeln, verwenden wir den folgende Code Style Guide: \url{https://source.android.com/source/code-style.html#java-language-rules}.

Zur Vereinfachung einer späteren Veröffentlichung achten wir von Anfang an auf die Android Launch Checkliste:\\ \url{http://developer.android.com/distribute/tools/launch-checklist.html}.

\subsection{Einschränkungen}
Die folgenden Empfehlungen aus dem Code Style Guide werden wir nicht übernehmen:

\begin{itemize}
\item \textbf{Jedes File hat zuoberst ein Copyright Statement.} \\ Der gesamte Code erhält eine Lizenz welche zentral abgelegt wird. Bei jedem File ein solches Statement einzufügen ist unnötig und wäre redundant.
\item \textbf{Statische Variablen beginnen mit s. Nicht-öffentliche, nicht-statische Variablen beginne mit m. Alle andern werden klein geschrieben.} \\ Diese Konvention ist unnötig, da man diese Informationen problemlos auch dem Code entnehmen kann. Moderne IDEs wie Android Studio unterstützen dabei zusätzlich durch farbliches Hervorheben. In diesem Projekt werden alle Variablen, ausser Konstanten, klein geschrieben. Konstanten werden vollständig aus Grossbuchstaben zusammengesetzt.
\end{itemize}

\section{UX Guidelines}
\label{sec:UX Guidelines}

Bei der Entwicklung des User Interfaces nutzen wir den Material Design Style Guide:
\url{https://developer.android.com/design/index.html}. Dieser Guide ist eigentlich für Android 5.0 ausgelegt, unsere Applikation muss jedoch von Android 4.3 an lauffähig sein. Daher müssen einige Einschränkungen, z.B. bei der Verwendung von Animationen, in Kauf genommen werden.
