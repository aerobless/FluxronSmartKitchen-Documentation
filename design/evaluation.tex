\section{Evaluation von Libraries}
\label{sec:Evaluation von Libraries}
Zur Unterstützung der gewählten Architektur sind Libraries zu bevorzugen. In diesem Kapitel sind die Evaluationsergebnisse beschrieben.

\subsection{Event Bus Library}
Es stehen folgende zwei Libraries zur Auswahl:
\begin{enumerate}
\item Otto - An event bus by Square
\item Greenrobot EventBus
\end{enumerate}
Diese sollen nun nachfolgend nach den durch die Architektur vorgegebenen Kriterien bewertet werden.

\begin{table}[H]
\begin{tabular}{|p{8cm}|c|c|}
 \hline 
\textbf{Kriterium} & \textbf{1: Otto} & \textbf{2: GreenRobot} 
\\ \hline

Instanzierung von mehreren Buses & x & x
\\ \hline

Ausführung in eigenem Thread & (x) & x
\\ \hline

Ausführung in \ac{UI}-Thread & x & x
\\ \hline

Ausführungsthread pro Subscriber wählbar &   & x
\\ \hline

Neustart des Bus & x & x
\\ \hline

Caching der letzten Events &   & x
\\ \hline

Asynchrone Events &   & x
\\ \hline

Eventproduzenten & x & 
\\ \hline

\end{tabular}
\caption{Bewertung Event Bus Libraries}
\end{table}

\subsubsection{Bewertung und Fazit}
Mit den nicht vorhandenen Kriterien \enquote{Ausführungsthread pro Subscriber wählbar} und \enquote{Asynchrone Event} fehlen der Library \enquote{Otto} zwei unverzichtbare Features. Da das einzige Feature, welches von GreenRobot nicht unterstützt wird, die Eventproduzenten sind (diese können durch Request/Response einfach ersetzt werden), kann der GreenRobot EventBus eingesetzt werden.

GreenRobot EventBus ist unter der Apache License \cite{apache_license} Version 2.0  lizenziert. Diese erfordert lediglich eine Kopie der Lizenz mit der gelieferten Software. Da keine Modifikationen an der eigentlichen Library gemacht wird, ist dies die einzige Einschränkung.