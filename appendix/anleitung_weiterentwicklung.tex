\section{Anleitung Weiterentwicklung}
\label{anleitung_weiterentwicklung}

Die nachfolgenden Anleitungen dienen dazu der Fluxron Solutions AG die Weiterentwicklung der Applikation zu erleichtern.

\subsection{EDS Files aktualisieren}
Wenn die Fluxron Geräte mit einer neuen Software Version neue Parameter erhalten muss das \enquote{.eds File} der jeweiligen Geräteklasse aktualisiert werden.
Die \enquote{.eds Files} befinden sich unter \code{/buildSrc/resources/ch.fluxron.generator/}.

Nachdem ein .eds File aktualisiert wurde muss die Applikation neu gebuildet werden so dass die Klassen \code{DeviceParameter}, \code{ParamManager} sowie das Ressource File \code{parameters.xml} neu generiert werden. Nach dem Build stehen die neuen Parameter auch via Autocompletion zur Verfügung.

\subsection{Neue Geräteklasse hinzufügen}
Um eine neue Geräteklasse hinzuzufügen muss zuerst dass entsprechende \enquote{.eds File} in den Ordner \code{/buildSrc/resources/ch.fluxron.generator/} kopiert werden. Danach kann in der Klasse \code{ParamGenerator} mit der Methode \code{loadParameter(..)} das neue .eds File angezogen werden. Zum Schluss sollte die Applikation neu gebuildet werden so dass die generierten Klassen und Ressource Files neu erstellt werden.

Für die neue Geräteklasse müssen nun die Layouts der 4 Fragments Status, Usage, Errors und Config erstellt werden. Dazu können die Layouts der bestehenden Geräteklassen kopiert und angepasst werden. Weitere Informationen zum Einbinden von Parametern bietet auch der Abschnitt \ref{sub:parameter_einbinden}.

Zum Schluss muss noch in den Backing-Klassen der Layouts die neue Geräteklasse angezogen werden. Die Backing-Klassen befinden sich unter \code{app/java/ch.fluxron.fluxronapp} \code{/ui/fragments}. In der entsprechenden Klasse muss in der Methode \code{onCreateView} ein zusätzlicher Fall eingebaut werden welcher das neue Layout inflated.

\subsection{Parameter in UI einbinden}
\label{sub:parameter_einbinden}
Parameter können in den 4 Fragments (Status, Usage, Errors, Config) der Gerätedetailansicht eingebunden werden. Die dazugehörigen Layouts befinden sich unter \code{app/res/layout/}.

Als erstes muss also das entsprechende Fragment Layout geöffnet werden. Darin kann dann ein neues UI Control eingefügt werden. Für veränderbare Parameter sollte \code{ParameterEditable} verwendet werden, unveränderliche Parameter können als \code{ParameterView} oder \code{TemperatureBar} dargestellt werden. Die UI Controls verfügen über ein \enquote{ParamName}-Property wo der gewünschte Parameter mittels Autocompletion gesetzt werden kann. \code{ParameterEditable} Controls verfügen zudem über ein \enquote{editableAccessLevel}-Property womit die Sichtbarkeit für die unterschiedlichen AccessLevels der User gesteuert werden kann.

Wenn das neue Control in einer bestehenden Liste von Parametern eingefügt worden ist, dann ist keine weitere Änderung mehr notwendig. Falls das Control ausserhalb eingebaut wurde muss es in der entsprechenden Backing-Klasse noch angezogen werden. Die Backing-Klassen befinden sich unter \code{app/java/ch.fluxron.fluxronapp/ui/fragments}. In der entsprechenden Klasse muss beim Eintreten eines \code{DeviceChanged} Events die Methode \code{handleDeviceChanged(..)} auf dem Control des Parameters aufgerufen werden. Falls ein AccessLevel gesetzt wurde muss ausserdem die Methode \code{handleAccessLevel(..)} beim Eintreten eines \code{AccessGranted} Events aufgerufen werden.