%Achtung: erscheint nur im privaten pdf
\section{Persönlicher Bericht}
\label{Persönlicher Bericht}

\subsection{Lessons Learned beider Autoren}
\label{sub:lessons_learned}

\begin{itemize}
\item{Mockups wurden nicht nachgeführt}
\item{UnitTest Abdeckung schwierig und daher tief}
\item{EventBus ist interessante Architektur für async-lastige Architekturen}
\item{EventBus setzte keine Weak-References ein, was zu Memory Leaks führte}
\item{Android Studio ist praxistauglich}
\item{Kleine Unsicherheit bei Verlust eines Git-Branches}
\item{Backlog war zu wenig strukturiert (einface Wiki-Page)}
\item{Backlog war dennoch hilfreich}
\item{Kickoff war sehr ergiebig}
\item{Kleine Issues (ca. 3h) führen zu extremer Schätzungsgenauigkeit}
\item{Enge Zusammenarbeit mit dem Kunden ist sehr hilfreich und motiviert}
\end{itemize}

\subsection{Bericht Konstantin Kayed}

Dieses Projekt war für mich sowohl herausfordernd, als auch sehr lehrreich und überaus spannend. Die Aufgabenstellung war breit gefächert und reichte von der Bluetooth-Kommunikation, über die Entwicklung eines soliden Grundgerüsts für die Anwendungslogik, bis hin zur Konzeption und Umsetzung einer modernen Benutzeroberfläche mit aktuellen Technologien.

Beginnend mit dem Kick-Off Anfang September, bot das Projekt einen Einblick in die Schnittstelle zwischen Software und Hardware. Diese Kombination war bereits einer der Gründe, warum ich mich für diese Arbeit mit höchster Priorität beworben hatten. Ich wollte ein Projekt, welches ein breites Spektrum an Herausforderungen und ein \enquote{sichtbares} Endergebnis hat.

Ebenfalls interessant war die Erfahrung, mit einem anderen Teammitglied als bei der SA zu arbeiten. Das Teamwork funktionierte perfekt und wir unterstützten uns gegenseitig im Fall von krankheitsbedingten Ausfällen oder terminlichen Verhinderungen. Ich entwickelte einen grossen Teil der Benutzeroberfläche und grundlegenden Architektur, wo hingegen Theo hervorragende Arbeit bei der Umsetzung der Bluetooth-Schnittstelle und der Geräte-UI leistete.

Rückblickend gab es natürlich auch verbesserungswürdige Punkte. Die meisten davon betreffen das Projekt und sind unter \ref{sub:lessons_learned} \enquote{Lessons Learned} aufgeführt. Es gibt aber auch Punkte, welche für mich persönlich Potenzial haben. So bin ich zum Beispiel nicht ganz mit meinem Arbeitsaufwand zu Beginn des Projektes zufrieden. Ich denke ich hätte anfangs mehr Zeit investieren können, um weitaus mehr für die Benutzeroberfläche zu tun. Den Zeitaufwand dafür habe ich etwas unterschätzt.

Auch gibt es einige Codepassagen, welche nicht meinem Verständnis von \enquote{gutem Code} entsprechen. Diese stehen häufig im Zusammenhang mit Android-Systemfunktionen oder UI-Implementationen. Dabei werden viele verschachtelte Callbacks und anonyme Implementationen verwendet. Hierbei meine ich, hätte ich mehr Zeit in die Code-Qualität als in das Aussehen der Benutzeroberfläche investieren sollen.

Obwohl ich mit einigen Punkten nicht vollständig zufrieden bin, ist dieses Projekt eine sehr positive Erfahrung gewesen. Die von uns eingesetzte agile Arbeitsweise erwies sich als nützlich und praktikabel und erleichterte mir die Arbeit von Zuhause oder Unterwegs.

Ich konnte viele wertvolle Erfahrungen im Bereich von User Interfaces, einem Bereich, in dem ich bisher weniger tätig war, sammeln. Es machte mir ungemein Spass, die Oberfläche zu entwerfen. Dabei bleiben mir vor allem die Mockups und wöchentlichen Skype-Sitzungen mit dem Kunden in guter Erinnerung. Auch das Erstellen der Grafiken und Bilder für die Dokumentation bereitete mir Freude.

Persönlich bin ich mit den Ergebnissen der Arbeit sehr zufrieden. Es ist toll zu sehen, wie innerhalb von 14 Wochen eine App aus dem Nichts entsteht und die Mitarbeiter der Firma Fluxron begeistern kann. Ich denke ich konnte dem Projekt einiges beisteuern und mindestens genauso viel lernen.

\subsection{Bericht Theodor Winter}

Am meisten motiviert hat mich bei diesem Projekt dass unser Ergebnis auch tatsächlich eingesetzt wird und nicht in irgendeiner Schublade verstaubt. Aber auch die Vielfalt an Teilaufgaben wie die Bluetooth-Kommunikation, das Umsetzen einer benutzerfreundlichen UI sowie die direkte Kommunikation mit dem Kunden  hat das Projekt sehr spannend und lehrreich gemacht. 

Nach dem Kick-Off im September haben wir uns auch nicht direkt auf das Programmieren des Architekturprototypen gestürzt sondern zuerst sorgfältig die Analyse- und Projektmanagementaufgaben abgearbeitet. Ich habe mich damals gefragt ob wir nicht zu viel Zeit dafür investieren. Es hat sich aber gezeigt dass wir damit sehr gut gefahren sind. Nachdem wir Anfang Oktober mit dem Architekturprototypen begannen, mussten wir bis kurz vor Schluss der Arbeit keine Dokumente mehr schreiben. So konnten wir uns vollständig auf die Programmierung konzentrieren. 

Das Teamwork mit Konstantin funktionierte sehr gut. Und auch die Aufgabenteilung hat sehr gut geklappt und hat nie zu Konflikten geführt. Zu Beginn der Arbeit haben wir eine Art Bottom-Up/Top-Down Vorgehensweise gewählt wobei ich mit der Entwicklung der Bluetooth Kommunikation begann und Konstantin sich um das Implementieren des Event Busses und der grundlegenden Architektur gekümmert hat. Im weiteren Verlauf des Projekts haben wir uns dann in der Mitte getroffen und anfallende Aufgaben agil verteilt.

Einige \enquote{Lessons Learned} die sowohl mich als auch Konstantin betreffen sind bereits unter \ref{sub:lessons_learned} aufgeführt. Für mich persönlich haben im Hinblick auf zukünftige Projekte vor allem folgende Punkte Potenzial. Zu Beginn habe ich das Backlog in unserem Wiki eher weniger genutzt und bei noch zu erledigenden Aufgaben \enquote{Todo} Kommentare im Code hinterlegt. Obwohl diese Kommentare zwar vom \ac{IDE} hervorgehoben werden hat sich dass als nicht sehr effizient erwiesen. Wenn man z.B. eine Aufgabe erledigt hat und eine neue in Angriff nehmen will ist es deutlich mühsamer im Code nach einer passenden \enquote{Todo}-Stelle zu suchen als einfach im Wiki nachzuschauen.

Auch die Codequalität ist nicht überall gleich hoch. Ich hätte gerne noch etwas Zeit aufgewandt um gewisse Stellen zu refactoren. Da wir jedoch nicht sehr viele Unit-Tests haben und ich so die Applikation nach grösseren Refactorings manuell hätte testen müssen war einerseits das Riskio etwas zu beschädigen als auch der Aufwand zu gross. Die begrenzte Anzahl Unit-Tests ist allerdings nicht Nachlässigkeit, sondern durch die Art der Applikation schlicht gegeben. Nur sehr wenige Klassen lassen sich sinnvoll einzeln testen. 

Aber trotz diesen einzelnen Punkten war das Projekt ein voller Erfolg. Die von uns erstelle Applikation hat am Testtag bei Fluxron im Amriswil nicht nur alle Systemtests erfüllt sondern auch die anwesenden Mitarbeiter begeistert. Auch persönlich bin ich mit den Ergebnissen der Arbeit sehr zufrieden. Ich habe viel Neues gelernt und mein bestehendes Wissen der Android Applikationsentwicklung stark ausgebaut.