%Achtung: erscheint nur im privaten pdf
\section{Persönlicher Bericht}
\label{Persönlicher Bericht}

\subsection{Lessons Learned}
\label{sub:lessons_learned}

\subsection{Bericht K. Kayed}

Dieses Projekt war für mich sowohl herausfordernd, als auch sehr lehrreich und überaus spannend. Die Aufgabenstellung war breit gefächert und reichte von der Bluetooth-Kommunikation, über die Entwicklung eines soliden Grundgerüsts für die Anwendungslogik, bis hin zur Konzeption und Umsetzung einer modernen Benutzeroberfläche mit aktuellen Technologien.

Beginnend mit dem Kick-Off Anfang September, bot das Projekt einen Einblick in die Schnittstelle zwischen Software und Hardware. Diese Kombination war bereits einer der Gründe, warum ich mich für diese Arbeit mit höchster Priorität beworben hatten. Ich wollte ein Projekt, welches ein breites Spektrum an Herausforderungen und ein \enquote{sichtbares} Endergebnis hat.

Ebenfalls interessant war die Erfahrung, mit einem anderen Teammitglied als bei der SA zu arbeiten. Das Teamwork funktionierte perfekt und wir unterstützten uns gegenseitig im Fall von krankheitsbedingten Ausfällen oder terminlichen Verhinderungen. Ich entwickelte einen grossen Teil der Benutzeroberfläche und grundlegenden Architektur, wo hingegen Theo hervorragende Arbeit bei der Umsetzung der Bluetooth-Schnittstelle und der Geräte-UI leistete.

Rückblickend gab es natürlich auch verbesserungswürdige Punkte. Die meisten davon betreffen das Projekt und sind unter \ref{sub:lessons_learned} \enquote{Lessons Learned} aufgeführt. Es gibt aber auch Punkte, welche für mich persönlich Potenzial haben. So bin ich zum Beispiel nicht ganz mit meinem Arbeitsaufwand zu Beginn des Projektes zufrieden. Ich denke ich hätte anfangs mehr Zeit investieren können, um weitaus mehr für die Benutzeroberfläche zu tun. Den Zeitaufwand dafür habe ich etwas unterschätzt.

Auch gibt es einige Codepassagen, welche nicht meinem Verständnis von \enquote{gutem Code} entsprechen. Diese stehen häufig im Zusammenhang mit Android-Systemfunktionen oder UI-Implementationen. Dabei werden viele verschachtelte Callbacks und anonyme Implementationen verwendet. Hierbei meine ich, hätte ich mehr Zeit in die Code-Qualität als in das Aussehen der Benutzeroberfläche investieren sollen.

Obwohl ich mit einigen Punkten nicht vollständig zufrieden bin, ist dieses Projekt eine sehr positive Erfahrung gewesen. Die von uns eingesetzte agile Arbeitsweise erwies sich als nützlich und praktikabel und erleichterte mir die Arbeit von Zuhause oder Unterwegs.

Ich konnte viele wertvolle Erfahrungen im Bereich von User Interfaces, einem Bereich, in dem ich bisher weniger tätig war, sammeln. Es machte mir ungemein Spass, die Oberfläche zu entwerfen. Dabei bleiben mir vor allem die Mockups und wöchentlichen Skype-Sitzungen mit dem Kunden in guter Erinnerung. Auch das Erstellen der Grafiken und Bilder für die Dokumentation bereitete mir Freude.

Persönlich bin ich mit den Ergebnissen der Arbeit sehr zufrieden. Es ist toll zu sehen, wie innerhalb von 14 Wochen eine App aus dem Nichts entsteht und die Mitarbeiter der Firma Fluxron begeistern kann. Ich denke ich konnte dem Projekt einiges beisteuern und mindestens genauso viel lernen.

\subsection{Bericht T. Winter}

