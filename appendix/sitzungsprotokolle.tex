%Achtung: erscheint nur im privaten pdf
\section{Sitzungsprotokolle}
\label{Sitzungsprotokolle}

Um ein Produkt zu entwickeln dass nicht nur die Anforderungen auf Papier erfüllt sondern auch den Vorstellungen des Kunden entspricht ist eine enge Zusammenarbeit mit dem Kunden unverzichtbar. Daher haben wir wöchentliche Skype-Meeting mit Herr Reichard von der Fluxron Solutions AG durchgeführt. Bei diesen Meetings wurde jeweils der aktuelle Stand der Arbeit gezeigt, Fragen zur genauen Funktionsweise der Fluxron Geräte gestellt und die Wünsche des Kunden abgeholt. Bei einzelnen Sitzungen wurden ausserdem Usability Tests durchgeführt um sicherzustellen dass die Anwendung auch problemlos vom Kunden bedient werden kann.

\begin{table}[H]
\begin{tabularx}{\textwidth}{ l | l | X}
\textbf{Datum}& \textbf{Zeit} & \textbf{Beschreibung}\\ \hline
09.09.2015 & 09:10 & Kick-Off Meeting bei Fluxron in Amriswil\\ \hline
15.09.2015 & 11:00 & Skype-Konferenz mit Hr. Reichard\\ \hline
22.09.2015 & 11:00 & Skype-Konferenz mit Hr. Reichard\\ \hline
29.09.2015 & 11:00 & Skype-Konferenz mit Hr. Reichard\\ \hline
06.10.2015 & 11:00 & Skype-Konferenz mit Hr. Reichard\\ \hline
13.10.2015 & 11:00 & Skype-Konferenz mit Hr. Reichard\\ \hline
20.10.2015 & 11:00 & Skype-Konferenz mit Hr. Reichard\\ \hline
27.10.2015 & 10:00 & Zwischenpräsentation an der \ac{HSR}\\ \hline
03.11.2015 & 11:00 & Skype-Konferenz mit Hr. Reichard\\ \hline
09.11.2015 & 11:00 & Skype-Konferenz mit Hr. Reichard\\ \hline
17.11.2015 & 11:00 & Skype-Konferenz mit Hr. Reichard\\ \hline
15.11.2015 & 11:00 & Skype-Konferenz mit Hr. Reichard\\
\end{tabularx}
\caption{Sitzungen Kunde}
\end{table}

Mit unserem Betreuer Herr Mehta haben wir ebenfalls wöchentliche Meetings durchgeführt. Auch bei diesen Meetings wurde der aktuelle Projektstand und Ausblick besprochen. Zudem gaben uns diese Meetings die Möglichkeit in Erfahrung zu bringen aus was wir bei der Dokumentation genau achten müssen.

\begin{table}[H]
\begin{tabularx}{\textwidth}{ l | l | X}
\textbf{Datum}& \textbf{Zeit} & \textbf{Beschreibung}\\ \hline
09.09.2015 & 09:10 & Kick-Off Meeting bei Fluxron in Amriswil\\ \hline
15.09.2015 & 10:00 & Teamsitzung mit Hr. Mehta\\ \hline
22.09.2015 & 10:00 & Teamsitzung mit Hr. Mehta\\ \hline
29.09.2015 & 10:00 & Teamsitzung mit Hr. Mehta\\ \hline
06.10.2015 & 10:00 & Teamsitzung mit Hr. Mehta\\ \hline
13.10.2015 & 10:00 & Teamsitzung mit Hr. Mehta\\ \hline
20.10.2015 & 10:00 & Teamsitzung mit Hr. Mehta\\ \hline
27.10.2015 & 10:00 & Zwischenpräsentation für den Experte\\ \hline
03.11.2015 & 14:00 & Zwischenpräsentation für den Gegenleser\\ \hline
09.11.2015 & 10:00 & Teamsitzung mit Hr. Mehta\\ \hline
17.11.2015 & 10:00 & Teamsitzung mit Hr. Mehta\\ \hline
01.12.2015 & 10:00 & Teamsitzung mit Hr. Mehta\\ \hline
08.12.2015 & 10:00 & Teamsitzung mit Hr. Mehta\\ \hline
15.11.2015 & 10:00 & Teamsitzung mit Hr. Mehta\\ \hline
\end{tabularx}
\caption{Sitzungen Betreuer}
\end{table}

\subsection{Transkripte der Sitzungen}
\label{sub:sitzungs_transkripte}

Nachfolgend sind die Transkripte aller Sitzungen in chronologischer Reihenfolge aufgeführt.

\begin{table}[H]
\begin{tabularx}{\textwidth}{| l | X |}
\hline
\textbf{Datum:} 09.09.2015
\textbf{Zeit:} 09:10
&
\textbf{Beschreibung:} Kick-Off in Amriswil \\ \hline
\specialcell[t]{
\textbf{Traktanden:}\\
- Gegenseitige Vorstellung\\
- Rahmenbedingungen\\
- Arbeits- und\\~ Kommunikationsablauf\\
- Material und Testgeräte\\
- Entwicklungs- und\\~ Testumgebung\\
- Arbeitsumfang und\\~ Aufgabenstellung\\}
& 
\specialcell[t]{
\textbf{Rahmenbedingungen:}\\
- 360 Stunden Arbeitszeit pro Person\\
- 50\,\% Projektleitung \& Dokumentation\\
- Projektdauer 14.09.2015-18.12.2015\\
\\
\textbf{Arbeits- und Kommunikationsablauf}\\
- Regelmässige Sitzungen am Dienstag morgen\\~ via Skype mit Hr. Reichard (Zeit noch offen)\\
- Zwischenpräsentation kurz nach Halbzeit\\
- Geheimhaltung noch abzuklären\\
- Abwesenheit von Hr. Reichard 23.11 - 11.12\\
\\
\textbf{Material und Testgeräte}\\
- Es wird auf einem Android Smartphone getestet\\
- Fluxron kann 1-2 Testgeräte pro Typ zur\\~ Verfügung stellen\\
- Minimalversion Android ist durch das\\~ Projektteam zu definieren\\
- CANopen Schnittstelle wird eingesetzt\\
\\
\textbf{Entwicklungs- und Testumgebung}\\
- Das Projektteam wählt eine angemessene\\~ Umgebung auf Basis von Android Studio\\
- Die bereits existierenden Apps sind mit\\~ Eclipse entwickelt worden\\
- Massentests könnten vor Ort in Amriswil\\~ gemacht werden, kleinere Tests mit den\\~ Testgeräten in Rapperswil an der HSR\\
\\
\textbf{Arbeitsumfang und Aufgabenstellung}\\
- Genutzt wird die Anwendung von Mitarbeitern\\~ von Fluxron und externen Servicetechnikern)\\
- Toleranz für kurzzeitig nicht erreichbare Geräte\\
- Die Benutzeroberfläche ist auf Englisch\\
}
\\ \hline
\end{tabularx}
\caption{Sitzung 09.09.2015 - Kick-Off in Amriswil}
\end{table}



\begin{table}[H]
\begin{tabularx}{\textwidth}{| l | X |}
\hline
\textbf{Datum:} 15.09.2015
\textbf{Zeit:} 10:00
&
\textbf{Beschreibung:} Teamsitzung mit Hr. Mehta \\ \hline
\specialcell[t]{
\textbf{Traktanden:}\\
- Aufgabenstellung abschliessen\\
- Meilensteine\\
- Android Version\\
- Arbeitsumgebung\\
}
& 
\specialcell[t]{
\textbf{Erkenntnisse:}\\
- Hr. Mehta erhält einen Account für die Seite\\
- Aufgabenstellung definitiv festgelegt\\
- Hr. Mehta klärt die Termine für die Abgabe\\~ des Berichtes\\
- Meilensteine sind ok\\
- Vorabklärung bezüglich Android Version nötig\\
- Hr. Mehta erhält Zugriff auf die Git-Repositories\\
- Status Zeitaufwände direkt aus Redmine\\
- Auch die Dokumentation (\LaTeX) wird mittels\\~ Continuous Integration generiert\\
- Externer Prüfungsexperte wird noch ermittelt\\
}
\\ \hline
\end{tabularx}
\caption{Sitzung 15.09.2015 - Teamsitzung mit Hr. Mehta}
\end{table}



\begin{table}[H]
\begin{tabularx}{\textwidth}{| l | X |}
\hline
\textbf{Datum:} 15.09.2015
\textbf{Zeit:} 11:00
&
\textbf{Beschreibung:} Skype-Konferenz mit Hr. Reichard \\ \hline
\specialcell[t]{
\textbf{Traktanden:}\\
- Testlauf Kommunikation\\
- Testgeräte\\
- Ausblick\\
}
& 
\specialcell[t]{
\textbf{Erkenntnisse:}\\
- Anrufen / Kommunikation funktioniert\\
- Auf Geheimhaltung muss nicht geachtet werden.\\
- Wir erhalten die folgenden Testgeräte:\\
~~- \textit{2x Bluetooth Classic (Thermostat, Induktion)}\\
~~- \textit{1x Bluetooth 4.0}\\
- Dokumentation Protokoll vorhanden\\
- PC Tool vorhanden\\
- App Source kann in Teilen rausgegeben werden\\
- Logo wird von Fluxron zugestellt\\
- Fluxron sendet Testgeräte per Post\\
- Nächste Woche wieder um 11 Uhr\\
- Zugang zu Redmine Server wird für\\~ Hr. Reichard eingerichtet\\
}
\\ \hline
\end{tabularx}
\caption{Sitzung 15.09.2015 - Skype-Konferenz mit Hr. Reichard}
\end{table}


\begin{table}[H]
\begin{tabularx}{\textwidth}{| l | X |}
\hline
\textbf{Datum:} 22.09.2015
\textbf{Zeit:} 10:00
&
\textbf{Beschreibung:} Teamsitzung mit Hr. Mehta \\ \hline
\specialcell[t]{
\textbf{Traktanden:}\\
- Projektplan\\
- Projektstand\\
- Analyse-Ergebnisse\\
- Endtermin der BA\\
}
& 
\specialcell[t]{
\textbf{Erkenntnisse:}\\
- Projektplan wurde vorgestellt\\
- Aktueller Projektstand wurde besprochen,\\~ wir sind auf Kurs\\
- Android Version 4.3 ist ermittelt worden\\
- Termin am 18.12.15 ist definitiv\\
- Hr. Mehta wählt einen externen Experten im\\~ Bereich UI/Anwendungsarchitektur (SE)\\
- UI-Konzept wird als Vorschlag an Kunde\\~ übermittelt\\
- Projektteam klärt mit Hr. Spielmann\\~ wegen Schliessfach für Testgeräte\\
}
\\ \hline
\end{tabularx}
\caption{Sitzung 22.09.2015 - Teamsitzung mit Hr. Mehta}
\end{table}


\begin{table}[H]
\begin{tabularx}{\textwidth}{| l | X |}
\hline
\textbf{Datum:} 22.09.2015
\textbf{Zeit:} 11:00
&
\textbf{Beschreibung:} Skype-Konferenz mit Hr. Reichard \\ \hline
\specialcell[t]{
\textbf{Traktanden:}\\
- Domainmodell\\
- NFR\\
- Informationen zur\\~ Dokumentation\\
}
& 
\specialcell[t]{
\textbf{Erkenntnisse:}\\
- NFR: 30 bis 50 Geräte, in einzelnen Fällen\\~ bis zu 150\\
- Inzwischen überarbeitete Dokumentation\\~ (mit den untersch. Geräteklassen) wird zugestellt\\
- Es gibt eine neue Version der App\\
- Geräte sind unterwegs per Post\\~ (2x CClass /Classic, 1x BT 4)\\
- Der BT4 Node ist nur mit seiner ID verfügbar,\\~ er hat ansonsten keine Funktionalität\\
- Nächster Termin: Nächster Di 11-12\\
- Minimale Android Version 4.3 ist bestätigt\\
}
\\ \hline
\end{tabularx}
\caption{Sitzung 22.09.2015 - Skype-Konferenz mit Hr. Reichard}
\end{table}


\begin{table}[H]
\begin{tabularx}{\textwidth}{| l | X |}
\hline
\textbf{Datum:} 29.09.2015
\textbf{Zeit:} 10:00
&
\textbf{Beschreibung:} Teamsitzung mit Hr. Mehta \\ \hline
\specialcell[t]{
\textbf{Traktanden:}\\
- Projektstand\\
- Analyseergebnisse\\
~~- \textit{Use Cases / User Stories}\\
~~- \textit{Testspezifikation}\\
- Testsystem Fluxron,\\~ Labor-Netzteil\\
- Ausblick\\
}
& 
\specialcell[t]{
\textbf{Erkenntnisse:}\\
- Name des Meilensteins \enquote{Inception} \\~ zu Elaboration umwandeln\\
- Analyse: Nummerierte Liste für User Stories\\~ (Eindeutige Referenz)\\
- Schreibfehler Kennwortschutz\\
- Begrifflichkeiten: Glossar und Prioritäten\\
- Muss/Soll/Kann -> Prio 1, Prio 2, Prio 3\\
- Testspezifikation -> Systemtest-Spezifikation\\~ (Referenzen auf Testobjekt) - > Coverage\\
- Minimal Anforderung Anzahl Geräte\\~ in Testspezifikation beschreiben\\
- Netzteil 24V Maximalstrom(A)\\
- Power Source mit Herr Reichard klären\\
- Massnahmen zu Guidelines\\
- Reviews zum Prüfen der Guidelines einplanen\\
- Themen für nächste Woche: Architektur,\\~ erster Prototyp und UI/UX, Domainmodell\\
}
\\ \hline
\end{tabularx}
\caption{Sitzung 29.09.2015 - Teamsitzung mit Hr. Mehta}
\end{table}



\begin{table}[H]
\begin{tabularx}{\textwidth}{| l | X |}
\hline
\textbf{Datum:} 29.09.2015
\textbf{Zeit:} 11:00
&
\textbf{Beschreibung:} Skype-Konferenz mit Hr. Reichard \\ \hline
\specialcell[t]{
\textbf{Traktanden:}\\
- Besprechung der Analyse\\
- User Stories\\
- Technische Anforderungen\\~ Testsystem\\
}
& 
\specialcell[t]{
\textbf{Erkenntnisse:}\\
- Netzteil: Hr. Reichard sendet ein Netzteil (24V)\\~ (Anzuschliessen an normale Steckdose und\\~ mittleren Stromstecker der Platinen)\\
- User Stories: Mit Herr Reichard besprochen\\
- Speichern der Küche explizit in Anforderungen\\~ erwähnen\\
- Verbindungsstatus soll klar ersichtlich sein\\
- Passwortschutz braucht es definitiv\\~ (Verschiedene Auth. Level Entwicklerpasswort,\\~ Level in EDS festlegen?)\\
- EDS-Files in App integrieren (Parameter mit\\~ Datentyp, eventuell auch Zugriffssteuerung),\\~ Hr. Reichard schickt uns EDS-Files\\
- Passwort zur Kommunikation mit Fluxron\\~ Geräten ist per Default \enquote{1234} und wird\\~ normalerweise nicht geändert\\
}
\\ \hline
\end{tabularx}
\caption{Sitzung 29.09.2015 - Skype-Konferenz mit Hr. Reichard}
\end{table}



\begin{table}[H]
\begin{tabularx}{\textwidth}{| l | X |}
\hline
\textbf{Datum:} 06.10.2015
\textbf{Zeit:} 10:00
&
\textbf{Beschreibung:} Teamsitzung mit Hr. Mehta \\ \hline
\specialcell[t]{
\textbf{Traktanden:}\\
- Projektstand\\
- Architektur\\
- Stand Prototyp\\
- UI: Farben und Navigation\\
- Domainmodell\\
- Ausblick\\
- Externer Prüfer\\
}
& 
\specialcell[t]{
\textbf{Erkenntnisse:}\\
- Projektstand\\
- Zeitbuchung gezeigt\\
- Darstellung Blackbox hat ein +Zeichen\\
- Besprechung Architektur und Evaluation\\~ Libraries\\
- Kurzvorstellung UI-Konzept\\
- Navigationsmodell: Verbindung von Layout\\
- Domainmodell: Attributmodifier ausblenden,\\~ Sensoren und Inputkomponenten\\
- Externer Prüfer: Evtl. Prüfung auf English\\
}
\\ \hline
\end{tabularx}
\caption{Sitzung 06.10.2015 - Teamsitzung mit Hr. Mehta}
\end{table}



\begin{table}[H]
\begin{tabularx}{\textwidth}{| l | X |}
\hline
\textbf{Datum:} 06.10.2015
\textbf{Zeit:} 11:00
&
\textbf{Beschreibung:} Skype-Konferenz mit Hr. Reichard \\ \hline
\specialcell[t]{
\textbf{Traktanden:}\\
- UI: Navigation und\\~ Farbschema\\
- Stand Dokumentation\\
- Netzteil erhalten\\
- diverses\\
}
& 
\specialcell[t]{
\textbf{Erkenntnisse:}\\
- Farbschema besprochen\\
- Navigationsmodell besprochen\\
- Optionales Feature: Erkennung der Küche\\~ anhand der BT-Geräte\\
- Information zu Zwischenpräsentation\\
- Dokumentation besprochen\\
}
\\ \hline
\end{tabularx}
\caption{Sitzung 06.10.2015 - Skype-Konferenz mit Hr. Reichard}
\end{table}



\begin{table}[H]
\begin{tabularx}{\textwidth}{| l | X |}
\hline
\textbf{Datum:} 13.10.2015
\textbf{Zeit:} 10:00
&
\textbf{Beschreibung:} Teamsitzung mit Hr. Mehta \\ \hline
\specialcell[t]{
\textbf{Traktanden:}\\
- Projektstand\\
- Stand Prototyp\\
- UI Mockups / Design\\
- Zwischenpräsentation:\\~ Inhalt, Ort, Zeitpunkt\\
- Ausblick\\
}
& 
\specialcell[t]{
\textbf{Erkenntnisse:}\\
- Traktanden auch per Mail zustellen\\
- Industrie 4.0: Tagung an der HSR,\\~ evtl. Projektvorstellung\\
- Zwischenpräsentation:\\
~~- \textit{Erster Entwurf der Doku., Demo, Architektur}\\
~~- \textit{Ort: Rapperswil, evtl. Mittagessen nachher}\\
~~- \textit{Sprache: Englisch}\\
~~- \textit{Dauer: 30min Präsentation 30min Diskussion}\\
- Ext. Prüfer: Swisscom Directories Lead Architect\\
- Prototyp kurz gezeigt\\
}
\\ \hline
\end{tabularx}
\caption{Sitzung 13.10.2015 - Teamsitzung mit Hr. Mehta}
\end{table}



\begin{table}[H]
\begin{tabularx}{\textwidth}{| l | X |}
\hline
\textbf{Datum:} 13.10.2015
\textbf{Zeit:} 11:00
&
\textbf{Beschreibung:} Skype-Konferenz mit Hr. Reichard \\ \hline
\specialcell[t]{
\textbf{Traktanden:}\\
- Bluetooth-Kommunikation\\
- Prototyp: Aktueller Stand\\
- UI Mockups\\
- Infos Zwischenpräsentation\\
}
& 
\specialcell[t]{
\textbf{Erkenntnisse:}\\
- Bluetooth: CANopen Dokument erhalten\\
~~- \textit{index + subindex}\\
~~- \textit{4byte + 2byte}\\
~~- \textit{Standard CANopen, Index und Subindex sind}\\~~ \textit{spezifisch für Fluxron}\\
\\
- UI Mockup - Übersicht:\\
~~- \textit{Foto ist eine gute Idee}\\
~~- \textit{Mehrere Fotos für mehrere Bereiche}\\~~ \textit{(Kartenstapel als Küchenübersicht)}\\
~~- \textit{Status "Disconnected" visuell darstellen}\\
\\
- UI Mockup - Status:\\
~~- \textit{Max/min Temperaturbereich wird vom}\\~~ \textit{Thermostat geliefert}\\
~~- \textit{Thermostat kann Programm 1-4 haben}\\~~ \textit{(Subindex 1-4)}\\
~~- \textit{Direktzugang zu Gerätestatus}\\
}
\\ \hline
\end{tabularx}
\caption{Sitzung 13.10.2015 - Skype-Konferenz mit Hr. Reichard}
\end{table}



\begin{table}[H]
\begin{tabularx}{\textwidth}{| l | X |}
\hline
\textbf{Datum:} 20.10.2015
\textbf{Zeit:} 10:00
&
\textbf{Beschreibung:} Teamsitzung mit Hr. Mehta \\ \hline
\specialcell[t]{
\textbf{Traktanden:}\\
- Projektstand\\
- Prototyp abgeschlossen\\
- Rückschau auf Dev letzter Woche\\
- Ausblick\\
- Wem Dokumentation zustellen?\\
- Beamer: Auflösung und Anschluss\\
}
& 
\specialcell[t]{
\textbf{Erkenntnisse:}\\
- Doku an Experten und Herrn Huser\\
- Prototyp gezeigt\\
- Beamer ausprobiert\\
- Stand gezeigt\\
}
\\ \hline
\end{tabularx}
\caption{Sitzung 20.10.2015 - Teamsitzung mit Hr. Mehta}
\end{table}



\begin{table}[H]
\begin{tabularx}{\textwidth}{| l | X |}
\hline
\textbf{Datum:} 27.10.2015
\textbf{Zeit:} 11:00
&
\textbf{Beschreibung:} Zwischenpräsentation an der HSR \\ \hline
\specialcell[t]{
\textbf{Teilnehmer:}\\
- F. Mehta (Betreuer)\\
- M. Reichard (Experte)\\
- V. Kriplaney (Fluxron AG)\\
}
& 
\specialcell[t]{
\textbf{Feedback \& Notizen:}\\
- Hr. Huser war nicht anwesend, es wird für ihn\\~ eine zweite, kurze - Zwischenpräsentation geben.\\~ Hr. Mehta wird Datum versenden.\\
- Evtl. Code-Review mit Hr. Kriplaney im Verlauf\\~ der zweiten Hälfte der Arbeit\\
- Evtl. Einsatz von Crash-Reporting Tool /\\~ Nachforschen wie das genau beim Play-Store läuft\\
- Präsentation verlief gut, alle Beteiligten waren\\~ beeindruckt\\
}
\\ \hline
\end{tabularx}
\caption{Sitzung 27.10.2015 - Zwischenpräsentation an der HSR}
\end{table}



\begin{table}[H]
\begin{tabularx}{\textwidth}{| l | X |}
\hline
\textbf{Datum:} 03.11.2015
\textbf{Zeit:} 14:00
&
\textbf{Beschreibung:} Teamsitzung mit Hr. Mehta und kurze Zwischenpräsentation für Hr. Huser\\ \hline
\specialcell[t]{
\textbf{Traktanden:}\\
- Zwischenpräsentation Gegenleser\\
- Projektstand\\
- Feedback Zwischenpräsentation\\
- Rückblick auf Iteration 2\\
- Umfang Dokumentation\\~ der Entwicklungsphase\\
- Ausblick\\
}
& 
\specialcell[t]{
\textbf{Erkenntnisse:}\\
- Projektstand gut\\
- Präsentation gut, evtl. Demo des Device an\\~ den Anfang nehmen da dann für Zuhörer\\~ das Konzept sofort klar wird.\\
- Dokumentation Implementation in Hinblick\\~ Erweiterbarkeit ist gut, generierte JavaDocs\\
- Nächste Woche schauen wir\\~ Bewertungsschema mit Hr. Mehta noch\\~ genauer an. Wenn wir Änderungen haben\\~ oder Fehler finden -> melden.\\
}
\\ \hline
\end{tabularx}
\caption{Sitzung 03.11.2015 - Zwischenpräsentation an der HSR}
\end{table}



\begin{table}[H]
\begin{tabularx}{\textwidth}{| l | X |}
\hline
\textbf{Datum:} 03.11.2015
\textbf{Zeit:} 11:00
&
\textbf{Beschreibung:} Skype-Konferenz mit Hr. Reichard \\ \hline
\specialcell[t]{
\textbf{Traktanden:}\\
- Parameter Geräte-Typ\\
- 1200 und 1200sub0\\~ unterscheidbar?\\
- EDS-Files für andere\\~ Device-Typen\\
- Aktueller Stand der App\\
- Pairingdialog\\
- Identify-Feature (Auto)\\
- UI-Entscheidungen zu:\\
~~- \textit{4 Thermostaten Programme}\\
~~- \textit{Back-Button raus}\\
~~- \textit{Küchenoptionen}\\
~~- \textit{Bearbeitungsmodus}\\
~~- \textit{Deviceoptionen}\\
}
& 
\specialcell[t]{
\textbf{Erkenntnisse:}\\
- Gerätetyp: 1018sub02 product code, wir\\~ erhalten Liste der Gerätetypen\\
- Parameter ohne Sub-Index: Hr. Reichard klärt\\~ ab, was der Grund dafür ist\\
- EDS-Files für Gerätetypen: ET, S und C haben\\~ eigene EDS-Files\\
- Information zu Pairing-Dialog mitgeteilt\\
- Welches ist der aktuelle Modus?\\~ Profil kann nur Hardwareseitig geändert werden.\\
- Display ansteuern? Testsequenz?\\~ Wir erhalten weitere Information\\
}
\\ \hline
\end{tabularx}
\caption{Sitzung 03.11.2015 - Skype-Konferenz mit Hr. Reichard}
\end{table}



\begin{table}[H]
\begin{tabularx}{\textwidth}{| l | X |}
\hline
\textbf{Datum:} 09.11.2015
\textbf{Zeit:} 10:00
&
\textbf{Beschreibung:} Teamsitzung mit Hr. Mehta \\ \hline
\specialcell[t]{
\textbf{Traktanden:}\\
- Projektstand\\
- Bewertungsraster\\
- Ausblick\\
}
& 
\specialcell[t]{
\textbf{Erkenntnisse:}\\
- Projektstand I.O.\\
- Bewertungsraster:\\
~~- \textit{\enquote{von Kunde bestätigt}: In Dokumentation evtl.}\\~~~ \textit{Tabelle mit Sitzungen, im Anhang}\\~~~ \textit{Sitzungsprotokolle.}\\
~~- \textit{\enquote{Analyse bestehender App}: Erwähnen das}\\~~~ \textit{proprietäres Protokoll daher unnötig zum}\\~~~ \textit{Vergleiche mit anderen Apps zu machen.}\\
~~- \textit{\enquote{Thema Accessibility}: muss vorkommen,}\\~~ \textit{darf aber recht kurz sein, erwähnen das wir}\\~~~ \textit{nicht nur Farben sondern z.B.}\\~~~ \textit{unterschiedliche Icons haben}\\
- Ausblick I.O.\\
}
\\ \hline
\end{tabularx}
\caption{Sitzung 09.11.2015 - Teamsitzung mit Hr. Mehta}
\end{table}



\begin{table}[H]
\begin{tabularx}{\textwidth}{| l | X |}
\hline
\textbf{Datum:} 09.11.2015
\textbf{Zeit:} 11:00
&
\textbf{Beschreibung:} Skype-Konferenz mit Hr. Reichard \\ \hline
\specialcell[t]{
\textbf{Traktanden:}\\
- Aktuellen Stand der App\\
- DeviceView: Parameter?\\
- Icon für die App\\
- Info / Copyright\\~ im Settings-Menü?\\
}
& 
\specialcell[t]{
\textbf{Erkenntnisse:}\\
- C-Klasse zuerst implementieren\\
- S und C Klasse sind sehr ähnlich\\
- Parameter aus bestehender App übernehmen,\\~ es müssen nicht alle Parameter sein, solange\\~ Beispiele vorhanden sind i.O. sie können es\\~ selbst erweitern\\
- Testen bei Fluxron während Testphase.\\~ Hr. Reichard ist weg von 23.11-11.12, d.h. haben\\~ andere Ansprechsperson\\
- Testen in echter Küche wäre eher schwer zu\\~ organisieren, aber sie würden bei\\ - Fluxron guten Testaufbau machen\\
- Icon soll wenn möglich unterschiedlich zu\\~ bestehendem aussehen dass man die Apps leicht\\~ auseinander halten kann. Frei wählbar von uns.\\~ Keine Priorität von Fluxron.\\
}
\\ \hline
\end{tabularx}
\caption{Sitzung 09.11.2015 - Skype-Konferenz mit Hr. Reichard}
\end{table}



\begin{table}[H]
\begin{tabularx}{\textwidth}{| l | X |}
\hline
\textbf{Datum:} 17.11.2015
\textbf{Zeit:} 10:00
&
\textbf{Beschreibung:} Teamsitzung mit Hr. Mehta \\ \hline
\specialcell[t]{
\textbf{Traktanden:}\\
- Projektstand\\
- Rückschau Entwicklung\\
- Ausblick\\
}
& 
\specialcell[t]{
\textbf{Erkenntnisse:}\\
- Projekt ist auf Kurs\\
- Vorabversion an Hr. Mehta, Hr. Huser und\\~ Hr. Kriplaney.
Code Walkthrough?\\
- Hr. Mehta klärt ob Industrie 4.0 interressiert ist\\
}
\\ \hline
\end{tabularx}
\caption{Sitzung 17.11.2015 - Teamsitzung mit Hr. Mehta}
\end{table}



\begin{table}[H]
\begin{tabularx}{\textwidth}{| l | X |}
\hline
\textbf{Datum:} 17.11.2015
\textbf{Zeit:} 11:00
&
\textbf{Beschreibung:} Skype-Konferenz mit Hr. Reichard \\ \hline
\specialcell[t]{
\textbf{Traktanden:}\\
- Projektstand\\
- Encoding der Betriebszähler\\
- Encoding der ErrorCodes\\
- Welcher Parameter zeigt\\~ Geräte-Klasse?\\
- Termin Testing bei Fluxron\\
- Hilfetexte für die Views\\
- Darstellung von Fehlern und\\~ Parameter ohne Inhalt\\
}
& 
\specialcell[t]{
\textbf{Erkenntnisse:}\\
- Betriebszähler: In Zehntelsstunden\\
- Error: Byte 3= Errornummer,\\~ Restliche Bytes = Power On Time\\
- Fehlerzeitpunkt: Betriebszähler ausgeben\\
- Evtl. Statusanzeige für Parameter laden/speichen\\
- Erkennung der Geräteklasse: Gemäss\\~ C/S in Bezeichnung\\~ LIFT=C, MIA weglassen, ET=ETX\\
- Dongle ist reiner Proxy => kein eigener Typ\\
- Fotos machen bei Tests\\
- Testtermin nächste Woche, Hr. Reichard klärt\\~ ab, ca. 10:00 bis Mittag, evtl. am Nachmittag\\~ noch eine Stunde\\
- Hilfetexte für Views hat nicht oberste Priorität,\\~ wenn dann einfache Text\\
- Leere Parameter: Antwort, wenn Parameter\\~ nicht existiert\\
- Schaltfläche für unpairing\\
}
\\ \hline
\end{tabularx}
\caption{Sitzung 17.11.2015 - Skype-Konferenz mit Hr. Reichard}
\end{table}



\begin{table}[H]
\begin{tabularx}{\textwidth}{| l | X |}
\hline
\textbf{Datum:} 01.12.2015
\textbf{Zeit:} 10:00
&
\textbf{Beschreibung:} Teamsitzung mit Hr. Mehta \\ \hline
\specialcell[t]{
\textbf{Traktanden:}\\
- Projektstand\\
- Resultat Testtag bei Fluxron\\
- Ausblick\\
}
& 
\specialcell[t]{
\textbf{Erkenntnisse:}\\
- Inhalt Abstract und Poster (Vorlagen und\\~ Vorgaben)\\
- Poster Ausgangslage, alte App nicht stark\\~ erwähnen, neue App vorstellen\\
- Abstract 200 Wörter, Kurzfassung der Arbeit\\
- Ablauf der Präsentation im Januar\\
- Länge der Präsentation: 30 min\\
- Mündl. Prüfung mit Fragen an das Team\\
~~- \textit{Arbeit, Architektur, Bedienung}\\
}
\\ \hline
\end{tabularx}
\caption{Sitzung 01.12.2015 - Teamsitzung mit Hr. Mehta}
\end{table}



\begin{table}[H]
\begin{tabularx}{\textwidth}{| l | X |}
\hline
\textbf{Datum:} 08.12.2015
\textbf{Zeit:} 10:00
&
\textbf{Beschreibung:} Teamsitzung mit Hr. Mehta \\ \hline
\specialcell[t]{
\textbf{Traktanden:}\\
- Projektstand\\
- App Demo\\
- Unterschrift Aufgabenstellung\\
- Abstract\\
- Lizenzvereinbarung?\\
- Anzahl gedruckte Exemplare\\
- Ausblick\\
}
& 
\specialcell[t]{
\textbf{Erkenntnisse:}\\
- App vorgeführt.\\
- Aufgabenstellung unterschrieben.\\
- Abstract Feedback:\\
~~- \textit{cloud-backend, klarer formulieren}\\
\textit{Positiver Abschlusssatz, Endnutzen}\\
- Lizenzvereinbarung?: Für kommerzielle Zwecke\\~ nutzbar durch Fluxron. Keine Vorlage der HSR.\\
- 5 gedruckte Exemplare, 4 cds\\
}
\\ \hline
\end{tabularx}
\caption{Sitzung 08.12.2015 - Teamsitzung mit Hr. Mehta}
\end{table}