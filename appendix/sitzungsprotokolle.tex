%Achtung: erscheint nur im privaten pdf
\chapter{Sitzungsprotokolle}
\label{Sitzungsprotokolle}

Um ein Produkt zu entwickeln dass nicht nur die Anforderungen auf Papier erfüllt sondern auch den Vorstellungen des Kunden entspricht ist eine enge Zusammenarbeit mit dem Kunden unverzichtbar. Daher haben wir wöchentliche Skype-Meeting mit Herr Reichard von der Fluxron Solutions AG durchgeführt. Bei diesen Meetings wurde jeweils der aktuelle Stand der Arbeit gezeigt, Fragen zur genauen Funktionsweise der Fluxron Geräte gestellt und die Wünsche des Kunden abgeholt. Bei einzelnen Sitzungen wurden ausserdem Usability Tests durchgeführt um sicherzustellen dass die Anwendung auch problemlos vom Kunden bedient werden kann.

\begin{table}[H]
\begin{tabularx}{\textwidth}{ l | l | X}
\textbf{Datum}& \textbf{Zeit} & \textbf{Beschreibung}\\ \hline
09.09.2015 & 09:10 & Kick-Off Meeting bei Fluxron in Amriswil\\ \hline
15.09.2015 & 11:00 & Skype-Konferenz mit Hr. Reichard\\ \hline
22.09.2015 & 11:00 & Skype-Konferenz mit Hr. Reichard\\ \hline
29.09.2015 & 11:00 & Skype-Konferenz mit Hr. Reichard\\ \hline
06.10.2015 & 11:00 & Skype-Konferenz mit Hr. Reichard\\ \hline
13.10.2015 & 11:00 & Skype-Konferenz mit Hr. Reichard\\ \hline
20.10.2015 & 11:00 & Skype-Konferenz mit Hr. Reichard\\ \hline
27.10.2015 & 10:00 & Zwischenpräsentation an der \ac{HSR}\\ \hline
03.11.2015 & 11:00 & Skype-Konferenz mit Hr. Reichard\\ \hline
09.11.2015 & 11:00 & Skype-Konferenz mit Hr. Reichard\\ \hline
17.11.2015 & 11:00 & Skype-Konferenz mit Hr. Reichard\\ \hline
15.11.2015 & 11:00 & Skype-Konferenz mit Hr. Reichard\\
\end{tabularx}
\caption{Sitzungen Kunde}
\end{table}

Mit unserem Betreuer Herr Mehta haben wir ebenfalls wöchentliche Meetings durchgeführt. Auch bei diesen Meetings wurde der aktuelle Projektstand und Ausblick besprochen. Zudem gaben uns diese Meetings die Möglichkeit in Erfahrung zu bringen aus was wir bei der Dokumentation genau achten müssen.

\begin{table}[H]
\begin{tabularx}{\textwidth}{ l | l | X}
\textbf{Datum}& \textbf{Zeit} & \textbf{Beschreibung}\\ \hline
09.09.2015 & 09:10 & Kick-Off Meeting bei Fluxron in Amriswil\\ \hline
15.09.2015 & 10:00 & Teamsitzung mit Hr. Mehta\\ \hline
22.09.2015 & 10:00 & Teamsitzung mit Hr. Mehta\\ \hline
29.09.2015 & 10:00 & Teamsitzung mit Hr. Mehta\\ \hline
06.10.2015 & 10:00 & Teamsitzung mit Hr. Mehta\\ \hline
13.10.2015 & 10:00 & Teamsitzung mit Hr. Mehta\\ \hline
20.10.2015 & 10:00 & Teamsitzung mit Hr. Mehta\\ \hline
27.10.2015 & 10:00 & Zwischenpräsentation für den Experte\\ \hline
03.11.2015 & 14:00 & Zwischenpräsentation für den Gegenleser\\ \hline
09.11.2015 & 10:00 & Teamsitzung mit Hr. Mehta\\ \hline
17.11.2015 & 10:00 & Teamsitzung mit Hr. Mehta\\ \hline
01.12.2015 & 10:00 & Teamsitzung mit Hr. Mehta\\ \hline
08.12.2015 & 10:00 & Teamsitzung mit Hr. Mehta\\ \hline
15.11.2015 & 10:00 & Teamsitzung mit Hr. Mehta\\ \hline
\end{tabularx}
\caption{Sitzungen Betreuer}
\end{table}

\subsection{Transkripte der Sitzungen}
\label{sub:sitzungs_transkripte}

Nachfolgend sind die Transkripte aller Sitzungen in chronologischer Reihenfolge aufgeführt.

\begin{table}[H]
\begin{tabularx}{\textwidth}{| l | X |}
\hline
\textbf{Datum:} 09.09.2015
\textbf{Zeit:} 09:10
&
\textbf{Beschreibung:} Kick-Off in Amriswil \\ \hline
\specialcell[t]{
\textbf{Traktanden:}\\
- Gegenseitige Vorstellung\\
- Rahmenbedingungen\\
- Arbeits- und\\~ Kommunikationsablauf\\
- Material und Testgeräte\\
- Entwicklungs- und\\~ Testumgebung\\
- Arbeitsumfang und\\~ Aufgabenstellung\\}
& 
\specialcell[t]{
\textbf{Rahmenbedingungen:}\\
- 360 Stunden Arbeitszeit pro Person\\
- 50\,\% Projektleitung \& Dokumentation\\
- 16 Wochen\\
- Start ab dem 14.09.2015\\

\textbf{Arbeits- und Kommunikationsablauf}\\
- Regelmässige Sitzungen am Dienstag morgen\\~ via Skype mit Hr. Reichard (Zeit noch offen)\\
- Zwischenpräsentation kurz nach Halbzeit\\
- Eventuelle Geheimhaltung einzelner Punkte \\~ muss geklärt werden durch Fluxron\\
- Abwesenheit von Hr. Reichard 23.11 - 11.12\\
\textbf{Material und Testgeräte}\\
- Es wird auf einem Android Smartphone getestet\\
- Fluxron kann diverse Testgeräte zur\\~ Verfügung stellen\\
- Ausreichend sind hierbei 1-2 jeden Typs\\
- Minimalversion Android ist durch das\\~ Projektteam zu definieren\\
- CANopen Schnittstelle wird eingesetzt\\
\textbf{Entwicklungs- und Testumgebung}\\
- Das Projektteam wählt eine angemessene\\~ Umgebung auf Basis von Android Studio\\
- Die bereits existierenden Apps sind mit\\~ Eclipse entwickelt worden, dieses wird allerdings\\~ nicht mehr von Google empfohlen\\
- Die Entwicklungsumgebung wird auf\\~ einem Server der HSR betrieben.\\
- Massentests könnten vor Ort in Amriswil\\ gemacht werden, kleinere Tests mit den Testgeräten\\~ in Rapperswil an der HSR\\

\textbf{Arbeitsumfang und Aufgabenstellung}\\
- Genutzt wird die Anwendung von Mitarbeitern\\~ von Fluxron und externen Servicetechnikern)\\
- Toleranz für kurzzeitig nicht erreichbare Geräte\\
- Die Benutzeroberfläche ist auf Englisch\\
}
\\ \hline
\end{tabularx}
\caption{Sitzung 09.09.2015 - Kick-Off in Amriswil}
\end{table}