% NFR

\section{Non Functional Requirements}
\label{sec:Non Functional Requirements}

\begin{enumerate}
\item Die Applikation soll 30 - 50 Fluxron Geräte problemlos verwalten können. In einzelnen Spezialfällen kann es aber bis zu 150 Geräte geben. Die App sollte damit ebenfalls umgehen können, allerdings sind leichte Einschränkungen kein Problem.
\item Die Applikation soll modular aufgebaut sein, so dass zukünftige Erweiterungen leicht realisiert werden können.
\item Der Code soll so dokumentiert werden, dass eine Weiterentwicklung der Applikation durch die \fluxron{} möglich ist.
\item Die Applikation soll auf Android Geräten die bis zu 2 Jahre alt sind lauffähig sein.
\item Sowohl Bluetooth Classic als auch Bluetooth 4.0 sollen von der Applikation unterstützt werden.
\item Die Applikation soll so einfach bedienbar sein, dass keine Schulung der Benutzer erforderlich ist.
\item Die Applikation soll auch ohne bestehende Internet-Verbindung vollständig funktionsfähig sein.
\end{enumerate}


\subsection{Anforderungen an Android-Version}
\label{subsec:Non Functional Requirements}
Aufgrund der alten Produktgeneration muss die App auf einer Android-Version laufen, welche klassische Bluetoothverbindungen unterstützt. Zudem sollte die App auch auf einer neuen Version von Android mit Bluetooth 4.0 \ac{LE} betrieben werden können.

Bluetooth 4.0 unterstützt sowohl klassische Bluetoothverbindungen, als auch die neuen Low Energy Verbindungen. Die klassischen Verbindungen sind rückwärtskompatibel mit den Vorgängerversionen.\cite{bt_standard}

Android unterstützt Bluetooth 4.0 (inklusive \ac{LE}) ab der API-Version 4.3\cite{bt_android}. Dies ist somit die minimal nötige Version, um die Kompatibilität zu gewährleisten.

Mit der Auswahl der Version 4.3 (Android JellyBean) können somit mehr als 50\% der momentan im Umlauf \cite{android_distribution} befindlichen Geräte unterstützt werden.