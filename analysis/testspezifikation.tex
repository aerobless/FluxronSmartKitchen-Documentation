\section{Testspezifikation}
\label{sec:Testspezifikation}

\subsection{Küche initialisieren}
\begin{table}[H]
\begin{tabularx}{\textwidth}{r X }
\textbf{Testziel} & Erfüllen von Use Case A. Der Benutzer erfasst die Lage der Geräte in der Küche. \\
\textbf{Vorbedingungen} & \begin{itemize}
\item Applikation gestartet
\item Geräte mit Bluetooth-Verbindung sind aktiv
\end{itemize} \\
\textbf{Durchführung} & \begin{enumerate}
\item Eine neue Küche erstellen und Namen vergeben
\item Das Layout der Küche nachstellen
\item Den Geräte-Suchlauf starten
\item Die gefundenen Geräte im Küchenlayout platzieren
\item Die Küchenerstellung abschliessen
\end{enumerate} \\
\textbf{Erwartetes Ergebnis} & In der Applikation ist die neu erstellte Küche sichtbar und die platzierten Geräte zeigen ihren Status an.\\
\end{tabularx}
\end{table}

\subsection{Küche anpassen}
\begin{table}[H]
\begin{tabularx}{\textwidth}{r X }
\textbf{Testziel} & Erfüllen von Use Case A. Der Benutzer kann die Initialisierung der Küche jederzeit wiederaufnehmen.\\
\textbf{Vorbedingungen} & \begin{itemize}
\item Applikation gestartet
\item Eine initialisierte Küche ist bereits in der Applikation vorhanden
\item Die in der initialisierten Küche verwendeten Geräte sind via Bluetooth erreichbar
\end{itemize} \\
\textbf{Durchführung} & \begin{enumerate}
\item Die bestehende Küche öffnen
\item Ein Gerät auswählen und entfernen
\item Den Geräte-Suchlauf starten
\item Das vorher entfernte Gerät soll gefunden werden und wieder zur Küche hinzugefügt werden
\item Die Küchenerstellung abschliessen
\end{enumerate} \\
\textbf{Erwartetes Ergebnis} & Die bestehende Küche hat nach der Durchführung dieses Tests die gleichen Geräte konfiguriert wie vorher.\\
\end{tabularx}
\end{table}

\subsection{Küche löschen}
\begin{table}[H]
\begin{tabularx}{\textwidth}{r X }
\textbf{Testziel} & Eine bereits erstelle Küche kann gelöscht werden.\\
\textbf{Vorbedingungen} & \begin{itemize}
\item Applikation gestartet
\item Eine initialisierte Küche ist bereits in der Applikation vorhanden
\end{itemize} \\
\textbf{Durchführung} & Eine bestehende Küche auswählen und löschen. \\
\textbf{Erwartetes Ergebnis} & Die gelöschte Küche ist nicht mehr in der Applikation sichtbar.\\
\end{tabularx}
\end{table}

\subsection{Protokoll auslesen}
\begin{table}[H]
\begin{tabularx}{\textwidth}{r X }
\textbf{Testziel} & Erfüllen von Use Case B. Der Benutzer kann das Protokoll eines Geräts auslesen. \\
\textbf{Vorbedingungen} & \begin{itemize}
\item Applikation gestartet
\item Eine initialisierte Küche ist bereits in der Applikation vorhanden
\item Die in der initialisierten Küche verwendeten Geräte sind via Bluetooth erreichbar
\end{itemize} \\
\textbf{Durchführung} & \begin{enumerate}
\item Die bestehende Küche öffnen
\item Ein Gerät auswählen und dessen Protokollspeicher auslesen
\end{enumerate} \\
\textbf{Erwartetes Ergebnis} & Das Protokoll eines in der Küche vorhandenen Geräts wurde gelesen.\\
\end{tabularx}
\end{table}

\subsection{Geräteparameter ändern}
\begin{table}[H]
\begin{tabularx}{\textwidth}{r X }
\textbf{Testziel} & Erfüllen von Use Case B2. Der Benutzer kann die Parameter eines Geräts verändern. \\
\textbf{Vorbedingungen} & \begin{itemize}
\item Applikation gestartet
\item Eine initialisierte Küche ist bereits in der Applikation vorhanden
\item Die in der initialisierten Küche verwendeten Geräte sind via Bluetooth erreichbar
\end{itemize} \\
\textbf{Durchführung} & \begin{enumerate}
\item Die bestehende Küche öffnen
\item Ein Gerät auswählen und die Parameterliste dieses Geräts anzeigen
\item Einen Parameter wählen und einen neuen, gültigen Wert eingeben
\item Speichern
\end{enumerate} \\
\textbf{Erwartetes Ergebnis} & Die Parameter eines Geräts wurden verändert.\\
\end{tabularx}
\end{table}

\subsection{Geräteparameter ändern mit Validierung}
\begin{table}[H]
\begin{tabularx}{\textwidth}{r X }
\textbf{Testziel} & Erfüllen von Use Case B2. Der Benutzer kann die Parameter eines Geräts verändern. \\
\textbf{Vorbedingungen} & \begin{itemize}
\item Applikation gestartet
\item Eine initialisierte Küche ist bereits in der Applikation vorhanden
\item Die in der initialisierten Küche verwendeten Geräte sind via Bluetooth erreichbar
\end{itemize} \\
\textbf{Durchführung} & \begin{enumerate}
\item Die bestehende Küche öffnen
\item Ein Gerät auswählen und die Parameterliste dieses Geräts anzeigen
\item Einen Parameter wählen und einen neuen, ungültigen Wert eingeben
\item Speichern
\end{enumerate} \\
\textbf{Erwartetes Ergebnis} & Der Parameter akzeptiert den ungültigen Wert nicht, es wird eine entsprechende Fehlermeldung dargestellt und die Änderung wird nicht gespeichert.\\
\end{tabularx}
\end{table}

\subsection{Kritische Geräteparameter ändern}
\begin{table}[H]
\begin{tabularx}{\textwidth}{r X }
\textbf{Testziel} & Erfüllen von Use Case C. Mitarbeiter von Fluxron können sicherheitsrelevante Geräteparameter verändern. \\
\textbf{Vorbedingungen} & \begin{itemize}
\item Eine initialisierte Küche ist bereits in der Applikation vorhanden
\item Die in der initialisierten Küche verwendeten Geräte sind via Bluetooth erreichbar
\end{itemize} \\
\textbf{Durchführung} & \begin{enumerate}
\item Die Applikation starten
\item Sich als Mitarbeiter der Fluxron identifizieren
\item Die bestehende Küche öffnen
\item Ein Gerät auswählen und dessen sicherheitsrelevanten Parameter anpassen
\end{enumerate} \\
\textbf{Erwartetes Ergebnis} & Ein sicherheitsrelevanter Parameter eines Geräts wurden verändert.\\
\end{tabularx}
\end{table}