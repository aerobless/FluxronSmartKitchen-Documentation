\pagebreak
\section[Systemtest Spezifikation]{Systemtest Spezifikation \footnote{Im Abschnitt \ref{sec:DetaillierteSystemtests} des Anhangs werden die Arbeitspakete detailliert aufgelistet.}}
\label{s:Systemtest_Spezifikation}
Zur Überprüfung der Applikationsfunktionalität werden Systemtests sowohl vom Entwicklerteam als, auch bei der \fluxron{} durchgeführt. 

\subsection{Systemtest Abdeckung}
\label{sub:systemtest_abdeckung}
Mit folgender Tabelle wird gezeigt das alle \ac{FR}, \ac{NFR} und Use Cases durch Systemtests abgedeckt sind.

\begin{table}[H]
\begin{tabularx}{\textwidth}{ l | l | l | X}
\textbf{\acs{UC}}& \textbf{\acs{FR}} & \textbf{\acs{NFR}} &\textbf{Systemtest}\\ \hline
A  & 3, 4 & & ST1: Küche initialisieren \\ \hline
A  & 5 & & ST2: Küche anpassen \\ \hline
A  &  & & ST3: Küche löschen \\ \hline
B  & 6, 11 & &  ST4: Protokoll auslesen \\ \hline
B2 & 6, 8, 9 & &  ST5: Geräteparameter ändern \\ \hline
B2 & 10 & & ST6: Geräteparameter ändern mit Validierung \\ \hline
C  &  & &  ST7: Kritische Geräteparameter ändern \\ \hline
   & 1 & &  ST8: Kennwortschutz  \\ \hline
   & 1 & & ST9: Kennwortschutz mit falschem Kennwort \\ \hline
A  & 2 & &  ST10: Suchlauf mit Gruppierung \\ \hline
   & 15 & &  ST11: Austausch von Küchenlayouts \\ \hline
   & & 1 &  ST12: Küche mit grosser Anzahl Geräte (50) \\ \hline
   & & 1 &  ST13: Küche mit grosser Anzahl Geräte (150) \\ \hline
   & & 5 &  ST14: Bluetooth Unterstützung \\ \hline
   & & 7 &  ST15: Keine Netzwerkkonnektivität \\
\end{tabularx}
\caption{Systemtest Abdeckung}
\end{table}
\vspace{-1cm}
\subparagraph{Nicht abgedeckte Anforderungen}
\begin{itemize}
\item \ac{FR} 16-17 (3. Priorität) fehlen in der Systemtest Abdeckungs Tabelle, da sie sich nur konzeptionell auf das Projekt auswirken und nicht getestet werden können.
\item \ac{NFR} 2-6 sind nicht durch Systemtests überprüfbar.
\end{itemize}

\subsection{Systemtest Durchführung}
\label{sub:systemtest_durchfuehrung}

Die Systemtests wurden im Verlauf der Arbeit nach jedem Meilenstein (siehe \ref{sec:Meilensteine}) durchgeführt, um den Fortschritt zu kontrollieren. Am 25.11.2015 wurde die Applikation bei Fluxron in Amriswil an voll funktionsfähigen Geräten getestet. Dabei wurden alle Systemtests erfüllt. Die Systemtests ST12 und ST13 konnten nur teilweise getestet werden, da keine Möglichkeit bestand, die Systemtests in einer echten Grossküche durchzuführen.

\begin{table}[H]
\begin{tabularx}{\textwidth}{ l | l | l | l | l | X }
\textbf{18.10} &\textbf{01.11} &\textbf{10.11} &\textbf{22.11} &\textbf{25.11}\footnotemark[1] &\textbf{Systemtest}\\ \hline
x & x & \checkmark & \checkmark & \checkmark &ST1: Küche initialisieren \\ \hline
x & x & \checkmark & \checkmark & \checkmark &ST2: Küche anpassen \\ \hline
\checkmark & \checkmark & \checkmark & \checkmark & \checkmark &ST3: Küche löschen \\ \hline
x & x & x & \checkmark & \checkmark &ST4: Protokoll auslesen \\ \hline
x & x & \checkmark & \checkmark & \checkmark &ST5: Geräteparameter ändern \\ \hline
x & x & x & \checkmark & \checkmark &ST6: Geräteparameter ändern mit Validierung \\ \hline
x & x & \checkmark & \checkmark & \checkmark &ST7: Kritische Geräteparameter ändern \\ \hline
x & x & x & x & \checkmark &ST8: Kennwortschutz  \\ \hline
x & x & x & x & \checkmark &ST9: Kennwortschutz mit falschem Kennwort \\ \hline
x & \checkmark & \checkmark & \checkmark & \checkmark &ST10: Suchlauf mit Gruppierung \\ \hline
x & x & x & \checkmark & \checkmark & ST11: Austausch von Küchenlayouts \\ \hline
x & x & \checkmark \footnotemark[2] & \checkmark \footnotemark[2] & \checkmark \footnotemark[2] &ST12: Küche mit grosser Anzahl Geräte (50) \\ \hline
x & x & \checkmark \footnotemark[2] &\checkmark \footnotemark[2] & \checkmark \footnotemark[2] &ST13: Küche mit grosser Anzahl Geräte (150) \\ \hline
\checkmark & \checkmark & \checkmark &  \checkmark &  \checkmark &ST14: Bluetooth Unterstützung \\ \hline
\checkmark & \checkmark & \checkmark & \checkmark &  \checkmark &ST15: Keine Netzwerkkonnektivität \\
\end{tabularx}
\caption{Systemtest Durchführungen}
\end{table}
\
\footnotetext[1]{Testtag bei Fluxron in Amriswil}
\footnotetext[2]{Getestet mit simulierten Geräten ohne echte Bluetooth Verbindung. Es bestand keine Möglichkeit in einer echten Küche mit 50-150 Geräten Tests durchzuführen.}