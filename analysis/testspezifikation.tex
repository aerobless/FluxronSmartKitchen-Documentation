\pagebreak
\section{Systemtest Spezifikation}
\label{s:Systemtest_Spezifikation}
Zur Überprüfung der Applikationsfunktionalität werden Systemtests sowohl vom Entwicklerteam als auch von der \fluxron{} durchgeführt.

\subsection{Systemtest Abdeckung}
\label{sub:systemtest_abdeckung}
Mit folgender Tabelle wird gezeigt das alle \ac{FR}, \ac{NFR} und Use Cases durch Systemtests abgedeckt sind.

\begin{table}[H]
\begin{tabularx}{\textwidth}{ l | l | l | X}
\textbf{\acs{UC}}& \textbf{\acs{FR}} & \textbf{\acs{NFR}} &\textbf{Systemtest}\\ \hline
A  & 3, 4 & & ST1: Küche initialisieren \\ \hline
A  & 5 & & ST2: Küche anpassen \\ \hline
A  &  & & ST3: Küche löschen \\ \hline
B  & 6, 11 & &  ST4: Protokoll auslesen \\ \hline
B2 & 6, 8, 9 & &  ST5: Geräteparameter ändern \\ \hline
B2 & 10 & & ST6: Geräteparameter ändern mit Validierung \\ \hline
C  &  & &  ST7: Kritische Geräteparameter ändern \\ \hline
   & 1 & &  ST8: Kennwortschutz  \\ \hline
   & 1 & & ST9: Kennwortschutz mit falschem Kennwort \\ \hline
A  & 2 & &  ST10: Suchlauf mit Gruppierung \\ \hline
   & 15 & &  ST11: Austausch von Küchenlayouts \\ \hline
   & & 1 &  ST12: Küche mit grosser Anzahl Geräte (50) \\ \hline
   & & 1 &  ST13: Küche mit grosser Anzahl Geräte (150) \\ \hline
   & & 5 &  ST14: Bluetooth Unterstützung \\ \hline
   & & 7 &  ST15: Keine Netzwerkkonnektivität \\
\end{tabularx}
\caption{Systemtest Abdeckung}
\end{table}

\subparagraph{Nicht abgedeckte Anforderungen}
\begin{itemize}
\item \ac{FR}16-17 (3. Priorität) fehlen in der Systemtest Abdeckungs Tabelle da sie sich nur konzeptionell auf das Projekt auswirken und nicht getestet werden können.
\item \ac{NFR}2-6 sind nicht durch Systemtests überprüfbar.
\end{itemize}

\subsection{Systemtest Durchführung}
\label{sub:systemtest_durchfuehrung}

Die Systemtests wurden im Verlauf der Arbeit nach jedem Meilenstein (siehe \ref{sec:Meilensteine}) durchgeführt um den Fortschritt zu kontrollieren. Am 25.11.2015 wurde die Applikation bei Fluxron in Amriswil an voll funktionsfähigen Geräten getestet. Dabei wurden alle Systemtests erfüllt. Einzig die Systemtests ST12 und ST13 konnten nicht vollständig getestet werden da keine Möglichkeit bestand die Systemtests in einer echten Grossküche durchzuführen.

\begin{table}[H]
\begin{tabularx}{\textwidth}{ l | l | l | l | l | X }
\textbf{18.10} &\textbf{01.11} &\textbf{10.11} &\textbf{22.11} &\textbf{25.11}\footnotemark[1] &\textbf{Systemtest}\\ \hline
x & x & \checkmark & \checkmark & \checkmark &ST1: Küche initialisieren \\ \hline
x & x & \checkmark & \checkmark & \checkmark &ST2: Küche anpassen \\ \hline
\checkmark & \checkmark & \checkmark & \checkmark & \checkmark &ST3: Küche löschen \\ \hline
x & x & x & \checkmark & \checkmark &ST4: Protokoll auslesen \\ \hline
x & x & \checkmark & \checkmark & \checkmark &ST5: Geräteparameter ändern \\ \hline
x & x & x & \checkmark & \checkmark &ST6: Geräteparameter ändern mit Validierung \\ \hline
x & x & \checkmark & \checkmark & \checkmark &ST7: Kritische Geräteparameter ändern \\ \hline
x & x & x & x & \checkmark &ST8: Kennwortschutz  \\ \hline
x & x & x & x & \checkmark &ST9: Kennwortschutz mit falschem Kennwort \\ \hline
x & \checkmark & \checkmark & \checkmark & \checkmark &ST10: Suchlauf mit Gruppierung \\ \hline
x & x & x & \checkmark & \checkmark & ST11: Austausch von Küchenlayouts \\ \hline
x & x & \checkmark \footnotemark[2] & \checkmark \footnotemark[2] & \checkmark \footnotemark[2] &ST12: Küche mit grosser Anzahl Geräte (50) \\ \hline
x & x & \checkmark \footnotemark[2] &\checkmark \footnotemark[2] & \checkmark \footnotemark[2] &ST13: Küche mit grosser Anzahl Geräte (150) \\ \hline
\checkmark & \checkmark & \checkmark &  \checkmark &  \checkmark &ST14: Bluetooth Unterstützung \\ \hline
\checkmark & \checkmark & \checkmark & \checkmark &  \checkmark &ST15: Keine Netzwerkkonnektivität \\
\end{tabularx}
\caption{Systemtest Durchführungen}
\end{table}
\
\footnotetext[1]{Testtag bei Fluxron in Amriswil}
\footnotetext[2]{Getestet mit simulierten Geräten ohne echte Bluetooth Verbindung. Es bestand keine Möglichkeit in einer echten Küche mit 50-150 Geräten Tests durchzuführen.}

\subsection{ST1: Küche initialisieren}
\begin{table}[H]
\begin{tabularx}{\textwidth}{r X }
\textbf{Testziel} & Erfüllen von Use Case A. Der Benutzer erfasst die Lage der Geräte in der Küche. \\
\textbf{Vorbedingungen} & \begin{itemize}
\item Applikation gestartet
\item Es sind mindestens 2 Geräte mit aktiver Bluetooth Verbindung vorhanden
\end{itemize} \\
\textbf{Durchführung} & \begin{enumerate}
\item Eine neue Küche erstellen und Namen vergeben
\item Das Layout der Küche nachstellen
\item Den Geräte-Suchlauf starten
\item Die gefundenen Geräte im Küchenlayout platzieren
\item Die Küchenerstellung abschliessen
\end{enumerate} \\
\textbf{Erwartetes Ergebnis} & In der Applikation ist die neu erstellte Küche sichtbar und die platzierten Geräte zeigen ihren Status an.\\
\end{tabularx}
\end{table}

\subsection{ST2: Küche anpassen}
\begin{table}[H]
\begin{tabularx}{\textwidth}{r X }
\textbf{Testziel} & Erfüllen von Use Case A. Der Benutzer kann die Initialisierung der Küche jederzeit wiederaufnehmen.\\
\textbf{Vorbedingungen} & \begin{itemize}
\item Applikation gestartet
\item Eine initialisierte Küche ist bereits in der Applikation vorhanden
\item Die Küche enthält mindestens 2 Geräte
\item Die verwendeten Geräte haben eine aktive Bluetooth Verbindung
\end{itemize} \\
\textbf{Durchführung} & \begin{enumerate}
\item Die bestehende Küche öffnen
\item Ein Gerät auswählen und entfernen
\item Den Geräte-Suchlauf starten
\item Das vorher entfernte Gerät soll gefunden werden und wieder zur Küche hinzugefügt werden
\item Die Küchenerstellung abschliessen
\end{enumerate} \\
\textbf{Erwartetes Ergebnis} & Die bestehende Küche hat nach der Durchführung dieses Tests die gleichen Geräte konfiguriert wie vorher.\\
\end{tabularx}
\end{table}

\subsection{ST3: Küche löschen}
\begin{table}[H]
\begin{tabularx}{\textwidth}{r X }
\textbf{Testziel} & Erfüllen von Use Case A. Eine bereits erstelle Küche kann gelöscht werden.\\
\textbf{Vorbedingungen} & \begin{itemize}
\item Applikation gestartet
\item Eine initialisierte Küche ist bereits in der Applikation vorhanden
\end{itemize} \\
\textbf{Durchführung} & Eine bestehende Küche auswählen und löschen. \\
\textbf{Erwartetes Ergebnis} & Die gelöschte Küche ist nicht mehr in der Applikation sichtbar.\\
\end{tabularx}
\end{table}

\subsection{ST4: Protokoll auslesen}
\begin{table}[H]
\begin{tabularx}{\textwidth}{r X }
\textbf{Testziel} & Erfüllen von Use Case B. Der Benutzer kann das Protokoll eines Geräts auslesen. \\
\textbf{Vorbedingungen} & \begin{itemize}
\item Applikation gestartet
\item Eine initialisierte Küche ist bereits in der Applikation vorhanden
\item Die Küche enthält mindestens ein Gerät welches via Bluetooth erreichbar ist
\end{itemize} \\
\textbf{Durchführung} & \begin{enumerate}
\item Die bestehende Küche öffnen
\item Ein Gerät auswählen und dessen Protokollspeicher auslesen
\end{enumerate} \\
\textbf{Erwartetes Ergebnis} & Das Protokoll eines in der Küche vorhandenen Geräts wurde gelesen.\\
\end{tabularx}
\end{table}

\subsection{ST5: Geräteparameter ändern}
\begin{table}[H]
\begin{tabularx}{\textwidth}{r X }
\textbf{Testziel} & Erfüllen von Use Case B2. Der Benutzer kann die Parameter eines Geräts verändern. \\
\textbf{Vorbedingungen} & \begin{itemize}
\item Applikation gestartet
\item Eine initialisierte Küche ist bereits in der Applikation vorhanden
\item Die Küche enthält mindestens ein Gerät welches via Bluetooth erreichbar ist
\end{itemize} \\
\textbf{Durchführung} & \begin{enumerate}
\item Die bestehende Küche öffnen
\item Ein Gerät auswählen und die Parameterliste dieses Geräts anzeigen
\item Einen Parameter wählen und einen neuen, gültigen Wert eingeben
\item Speichern
\end{enumerate} \\
\textbf{Erwartetes Ergebnis} & Die Parameter eines Geräts wurden verändert.\\
\end{tabularx}
\end{table}

\subsection{ST6: Geräteparameter ändern mit Validierung}
\begin{table}[H]
\begin{tabularx}{\textwidth}{r X }
\textbf{Testziel} & Erfüllen von Use Case B2. Der Benutzer kann die Parameter eines Geräts verändern. \\
\textbf{Vorbedingungen} & \begin{itemize}
\item Applikation gestartet
\item Eine initialisierte Küche ist bereits in der Applikation vorhanden
\item Die Küche enthält mindestens ein Gerät welches via Bluetooth erreichbar ist
\end{itemize} \\
\textbf{Durchführung} & \begin{enumerate}
\item Die bestehende Küche öffnen
\item Ein Gerät auswählen und die Parameterliste dieses Geräts anzeigen
\item Einen Parameter wählen und einen neuen, ungültigen Wert eingeben
\item Speichern
\end{enumerate} \\
\textbf{Erwartetes Ergebnis} & Der Parameter akzeptiert den ungültigen Wert nicht, es wird eine entsprechende Fehlermeldung dargestellt und die Änderung wird nicht gespeichert.\\
\end{tabularx}
\end{table}

\subsection{ST7: Kritische Geräteparameter ändern}
\begin{table}[H]
\begin{tabularx}{\textwidth}{r X }
\textbf{Testziel} & Erfüllen von Use Case C. Mitarbeiter von Fluxron können sicherheitsrelevante Geräteparameter verändern. \\
\textbf{Vorbedingungen} & \begin{itemize}
\item Eine initialisierte Küche ist bereits in der Applikation vorhanden
\item Die Küche enthält mindestens ein Gerät welches via Bluetooth erreichbar ist
\end{itemize} \\
\textbf{Durchführung} & \begin{enumerate}
\item Die Applikation starten
\item Sich als Mitarbeiter der Fluxron identifizieren
\item Die bestehende Küche öffnen
\item Ein Gerät auswählen und dessen sicherheitsrelevanten Parameter anpassen
\end{enumerate} \\
\textbf{Erwartetes Ergebnis} & Ein sicherheitsrelevanter Parameter eines Geräts wurden verändert.\\
\end{tabularx}
\end{table}

\subsection{ST8: Kennwortschutz}
\begin{table}[H]
\begin{tabularx}{\textwidth}{r X }
\textbf{Testziel} & Erfüllen von \ac{FR}1. Nur autorisierte Personen können die Applikation verwenden.\\
\textbf{Vorbedingungen} & Die Applikation ist installiert.\\
\textbf{Durchführung} & \begin{enumerate}
\item Die Applikation starten
\item Passwort eingeben
\end{enumerate} \\
\textbf{Erwartetes Ergebnis} & Die Applikation lässt sich nach Eingabe des Passworts normal verwenden.\\
\end{tabularx}
\end{table}

\subsection{ST9: Kennwortschutz mit falschem Kennwort}
\begin{table}[H]
\begin{tabularx}{\textwidth}{r X }
\textbf{Testziel} & Erfüllen von \ac{FR}1. Nur autorisierte Personen können die Applikation verwenden.\\
\textbf{Vorbedingungen} & Die Applikation ist installiert.\\
\textbf{Durchführung} & \begin{enumerate}
\item Die Applikation starten
\item Falsches Passwort eingeben
\end{enumerate} \\
\textbf{Erwartetes Ergebnis} & Die Applikation zeigt eine Fehlermeldung und erlaubt eine erneute Eingabe des Passworts.\\
\end{tabularx}
\end{table}

\subsection{ST10: Suchlauf mit Gruppierung}
\begin{table}[H]
\begin{tabularx}{\textwidth}{r X }
\textbf{Testziel} & Erfüllen von \ac{FR}2. Beim Suchlauf werden die gefundenen Geräte gruppiert dargestellt.\\
\textbf{Vorbedingungen} & \begin{itemize}
\item Applikation gestartet
\item Es sind mindestens 3 Geräte mit aktiver Bluetooth Verbindung vorhanden
\end{itemize} \\
\textbf{Durchführung} & Ein Geräte-Suchlauf durchführen\\
\textbf{Erwartetes Ergebnis} & Die gefundenen Geräte werden übersichtlich gruppiert dargestellt.\\
\end{tabularx}
\end{table}

%TODO muss noch angepasst werden wenn genauere Details verfügbar sind wie das Sharing funktionieren soll
\subsection{ST11: Austausch von Küchenlayouts}
\begin{table}[H]
\begin{tabularx}{\textwidth}{r X }
\textbf{Testziel} & Erfüllen von \ac{FR}15. Bereits initialisierte Küchen Konfigurationen können mit anderen Benutzern ausgetauscht werden. \\
\textbf{Vorbedingungen} & \begin{itemize}
\item Ein Android Handy mit einer bereits initialisierten Küche
\item Ein Android Handy ohne initialisierte Küchen
\end{itemize} \\
\textbf{Durchführung} & \begin{enumerate}
\item Auf einem Gerät die bestehende Küche öffnen
\item Die Küche mittels \enquote{Share-Button} teilen
%TODO Klären ob E-Mail Adresse, Benutzer-ID oder teilen via Bluetooth? 
\item Auf dem zweiten Gerät kommt eine Benachrichtigung dass eine neue Küchenkonfiguration verfügbar ist
\item Diese Konfiguration akzeptieren und öffnen
\end{enumerate} \\
\textbf{Erwartetes Ergebnis} & Die Küchenkonfiguration vom einen Gerät ist nun auch auf dem zweiten Gerät verfügbar.\\
\end{tabularx}
\end{table}

\subsection{ST12: Küche mit grosser Anzahl Geräte (50)}
\begin{table}[H]
\begin{tabularx}{\textwidth}{r X }
\textbf{Testziel} & Erfüllen von \ac{NFR}1. Die Applikation soll zwischen 30-50 Fluxron Geräte problemlos verwalten können. \\
\textbf{Vorbedingungen} & \begin{itemize}
\item Applikation gestartet
\item Es sind 50 Geräte (simuliert oder real) mit aktiver Bluetooth Verbindung vorhanden
\end{itemize} \\
\textbf{Durchführung} & \begin{enumerate}
\item Eine neue Küche initialisieren
\item 24 Geräte via Suchlauf finden und im Küchenlayout platzieren
\item Ein 25. Gerät hinzufügen und sich dessen Position/Name merke
\item Die restlichen 25 Geräte hinzufügen, so dass in der Küche jetzt total 50 Geräte vorhanden sind
\item die Küchen speichern
\item Gerät 25 in der Küche finden und dessen Status ansehen
\item Ein Parameter von Gerät 25 anpassen
\end{enumerate} \\
\textbf{Erwartetes Ergebnis} & Auch bei 50 Geräten sind auf einem spezifischen Gerät in der Küche noch alle Interaktionen möglich. Gerät 25 kann normal bedient werden.\\
\end{tabularx}
\end{table}

\subsection{ST13: Küche mit grosser Anzahl Geräte (150)}
\begin{table}[H]
\begin{tabularx}{\textwidth}{r X }
\textbf{Testziel} & Erfüllen von \ac{NFR}1. Die Applikation soll mit Einschränkungen bis zu 150 Fluxron Geräte verwalten können.\\
\textbf{Vorbedingungen} & \begin{itemize}
\item Applikation gestartet
\item Es sind 50 Geräte (simuliert oder real) mit aktiver Bluetooth Verbindung vorhanden
\end{itemize} \\
\textbf{Durchführung} & \begin{enumerate}
\item Eine neue Küche initialisieren
\item 150 Geräte via Suchlauf finden und im Küchenlayout platzieren
\item Prüfen dass der Status der Geräte sichtbar ist
\item Prüfen dass einzelne Geräte noch auswählbar sind und deren Parameter verändert werden können
\end{enumerate} \\
\textbf{Erwartetes Ergebnis} & Bei 150 Geräten sind mit Einschränkungen %TODO Einschränkungen genauer spzeifizieren
zu rechnen. Die Applikation hat aber keine Performance-Probleme und es können alle kritischen Funktionen verwendet werden.\\
\end{tabularx}
\end{table}

\subsection{ST14: Bluetooth Unterstützung}
\begin{table}[H]
\begin{tabularx}{\textwidth}{r X }
\textbf{Testziel} & Erfüllen von \ac{NFR}5. Die Applikation kann sowohl mit Bluetooth Classic als auch Bluetooth 4.0 Geräten kommunizieren. \\
\textbf{Vorbedingungen} & \begin{itemize}
\item Aktives Fluxron Blutooth Classic Gerät
\item Aktives Fluxron Blutooth 4.0 Gerät
\end{itemize}\\
\textbf{Durchführung} & \begin{enumerate}
\item Eine neue Küche initialisieren und Suchlauf starten
\item Die beiden Geräte hinzufügen
\end{enumerate} \\
\textbf{Erwartetes Ergebnis} & Der Status beider Geräte ist in der Küche sichtbar.\\
\end{tabularx}
\end{table}

\subsection{ST15: Keine Netzwerkkonnektivität}
\begin{table}[H]
\begin{tabularx}{\textwidth}{r X }
\textbf{Testziel} & Erfüllen von \ac{NFR}7. Die Applikation soll auch ohne aktive Internetverbindung voll funktionsfähig sein. \\
\textbf{Vorbedingungen} & \begin{itemize}
\item Applikation gestartet
\item WLAN, GSM/UMTS/LTE deaktiviert
\item Mindestens ein Gerät mit aktiver Bluetooth Verbindung vorhanden
\end{itemize}\\
\textbf{Durchführung} & \begin{enumerate}
\item Eine neue Küche initialisieren und Suchlauf starten
\item Das gefundene Gerät hinzufügen
\item Den Status des Geräts abrufen
\item Das Protokoll des Geräts abrufen
\item Die Küche löschen
\end{enumerate} \\
\textbf{Erwartetes Ergebnis} & Alle Aktionen sind möglich. Die Applikation verhält sich normal.\\
\end{tabularx}
\end{table}

