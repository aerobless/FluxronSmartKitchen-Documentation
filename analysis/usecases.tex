% Anforderungsanalyse

\section{User Stories}
\label{sec:User Stories}

\subsection{Höchste Priorität}
\label{subsec:Höchste Priorität}

\begin{itemize}
\item Als Servicetechniker oder Mitarbeiter der Fluxron möchte ich alle über Bluetooth erreichbaren Geräte nach Typ gruppiert mit einem Suchlauf finden können.
\item Als Servicetechniker möchte ich eine übersichtliche Darstellung der Küche und ihrer Geräte erfassen und beim Wartungstermin nutzen können. Damit kann ich die Geräte leicht wiederfinden.
\item Es kann vorkommen dass manche Geräte zwischenzeitlich nicht erreichbar sind. Als Techniker möchte ich diese Geräte dennoch auf der Übersicht sehen, will aber klar erkennen dass keine Verbindung besteht.
\item Als Servicetechniker möchte ich vor Ort Parameter und Protokolle der Geräte auslesen können. Wenn ich eine Einstellung verändern möchte, soll dies ebenfalls über die App funktionieren.
\item Als Servicetechniker möchte ich jederzeit den Überblick über den Status aller Geräte behalten können.
\item Als Entwickler von Fluxron möchte ich eine einfache Erweiterbarkeit des Programmcodes, damit ich die App später leicht anpassen kann.
\end{itemize}

\subsection{Zweite Priorität}
\label{subsec:Zweite Priorität}

\begin{itemize}
\item Als Mitarbeiter von Fluxron möchte ich die neue Gerätegeneration mit Bluetooth 4.0 ebenfalls ansprechen können.
\item Als angestellter Servicetechniker in einer Firma möchte ich die Küchenkonfiguration und das Layout mit meinem Mitarbeiter austauschen oder sichern können.
\end{itemize}

\subsection{Dritte Priorität}
\label{subsec:Dritte Priorität}

\begin{itemize}
\item Als Entwickler von Fluxron möchte ich zu einem späteren Zeitpunkt eine Internetanbindung für die App realisieren. (Konzeptionelle Unterstützung bereits jetzt in der App)
\item Als Entwickler von Fluxron möchte ich zu einem späteren Zeitpunkt automatische Fehlermeldungen von meinem Produkt über das Internet erhalten. (Konzeptionelle Unterstützung bereits jetzt in der App)
\item Als Servicetechniker möchte ich über das Internet informiert werden, wenn ein von mir Installiertes Gerät einen Fehler auslöst. (Konzeptionelle Unterstützung bereits jetzt in der App)
\item Als Techniker von Fluxron möchte ich zu einem späteren Zeitpunkt automatische Protokolle zu Nutzung und Verschleiss sehen können. (Konzeptionelle Unterstützung bereits jetzt in der App)
\item Als Techniker von Fluxron möchte ich zu einem späteren Zeitpunkt automatische Protokolle zu den Parameter- und Messwerten sehen. (Konzeptionelle Unterstützung bereits jetzt in der App)
\end{itemize}

\section{Use Cases}
\label{sec:Use Cases}

Im nachfolgenden Kapitel werden die primären Anwendungsfälle aufgelistet und beschrieben.