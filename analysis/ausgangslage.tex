% Ausgangslage (Ist-Analyse)

\section{Ausgangslage}
\label{sec:Ausgangslage}

\subsection{Anwendungsumfeld}
\label{subsec:Anwendungsumfeld}

\begin{wrapfigure}{l}{6cm}
	\includegraphics[scale=0.4]{analysis/res/einbaugeraete}
	\caption{Induktionsherd}
\end{wrapfigure}

Die Firma Fluxron AG mit Sitz in Amriswil TG bietet induktive Heiz- und Energiesysteme für Grossküchen an. Die Produktpalette besteht aus verschiedenen Induktionsherden und Thermostatsystemen. Diese haben jeweils ein Bluetoothmodul eingebaut, welches auf ein CANopen-basiertes Protokoll zur Kommunikation setzt. Die Module liefern neben den Geräteeinstellungen auch Fehlercodes und Sensormesswerte via Bluetooth.

Fluxron verkauft diese Induktionsgeräte an Servicefirmen, welche diese dann beim Endkunden, z.B. einem Restaurant, einbauen und installieren. Dabei, aber auch bei Wartungsarbeiten, setzen die Servicefirmen die bestehende Android Applikation \enquote{FLX Tool} zur Diagnose und zur Konfiguration ein.

Neben den Servicefirmen setzen auch die Mitarbeiter der Firma Fluxron die Androidapplikation bei internen Versuchen oder zur Ferndiagnose ein. Bei der Ferndiagnose wird eine Teamviewer-Verbindung auf das Smartphone oder den PC des Technikers aufgebaut. Über diese kann dann eine Diagnose mittels den Tools via Bluetooth erfolgen.

\subsection{Bestehende Android-Applikation - Fluxron Systemkonfigurator }
\label{subsec:Bestehende Smartphone-Applikation}
Fluxron hat in Eigenentwicklung bereits eine einfache Android-Applikation geschrieben. Diese unterstützt das Suchen von Geräten sowie das Lesen und Schreiben von Parametern. Da diese App allerdings immer nur ein Gerät gleichzeitig bedienen kann und die Benutzerinteraktion daher sehr zeitaufwändig ist, soll im Rahmen dieses Projektes eine neue, übersichtlichere Androidapplikation entstehen. In den nachfolgenden Abschnitten sind die Funktionen dieser Applikation aufgelistet.

\subsubsection{Installation und Konfiguration}
\label{subsubsec:Installation und Konfiguration}
Die App ist öffentlich im Google Play Store erhältlich. Nach dem Download muss der Benutzer aber ein Kennwort eingeben, um Zugang zur App und den Gerätefunktionen zu erhalten.

\begin{wrapfigure}[9]{l}{6cm}
\vspace{-1cm}
	\includegraphics[scale=0.3]{analysis/res/SelectDevice}
	\caption{Geräteauswahl FLX Tool}
\end{wrapfigure}

\subsubsection{Verbindungsaufbau zu einem Gerät}
\label{subsubsec:Verbindungsaufbau zu einem Gerät}

Um eine Verbindung mit einem Gerät herzustellen, muss dieses zuerst über einen Suchlauf gefunden werden. Der Suchlauf zeigt alle momentan aktiven Geräte in einer Liste mit ihrer Device-ID an. In dieser Liste sind neben den aktiven Geräten auch alle Geräte aufgelistet, welche schon einmal mit der App verbunden waren.

Danach kann man sich mit einem Gerät aus der Liste verbinden um mit dem Gerät zu interagieren.
\vspace{16pt}

\begin{wrapfigure}[9]{l}{6cm}
	\includegraphics[scale=0.3]{analysis/res/ShowStatusThermostat}
	\caption{Statusansicht \\Thermostat}
\end{wrapfigure}

\subsubsection{Setupansicht}
\label{subsubsec:Setupansicht}
Wenn ein Gerät verbunden ist, kann in der Setup-ansicht eine Liste aller Parameter des Gerätes ausgelesen werden. Der Benutzer kann Parameter anwählen, eine Beschreibung ansehen und Werte ändern.

\subsubsection{Historyansicht}
\label{subsubsec:Ansichten}
In der Historyansicht können Betriebszähler eingesehen werden. Sie enthält Werte wie die Anzahl von \enquote{Power-On} Events oder Laufzeit des Gerätes.

Neben den Zählern kann ein Fehlerlog eingesehen werden, welches die letzten zehn Fehlercodes zeigt. Dies wird zur Diagnose von Problemen genutzt.

\begin{wrapfigure}[8]{l}{6cm}
	\vspace{-0.5cm}
	\includegraphics[scale=0.3]{analysis/res/ShowStatusInduction}
	\caption{Statusansicht für \\Induktionsherd}
\end{wrapfigure}

\subsubsection{Einschränkungen der App}
\label{subsubsec:Einschränkungen der App}
\begin{itemize}
\item Es kann immer nur ein Gerät gleichzeitig verbunden sein.
\item Für jede Verbindung ist ein erneuter Suchlauf nötig.
\item Die Geräte sind nur mit der ID sichtbar, was die Orientierung in einer Grossküche mit einem dutzend Geräten schwierig macht.
\item Wenn ein nicht verbundenes Gerät eine Fehlermeldung absetzt, wird diese nicht empfangen.
\item Der Typ der verbundenen Geräte wird zwar erkannt, die Benutzeroberfläche bietet aber immer noch alle Gerätetypen an.
\item Wenn keine Geräte vorhanden sind, oder gefunden werden, hat die App keinen Nutzen für den User.
\end{itemize}

\WFclear
\subsection{Weitere bestehende Software}
\label{subsec:Weitere bestehende Software}

\begin{wrapfigure}[7]{l}{6cm}
\includegraphics[scale=0.4]{analysis/res/flxaccess}
	\caption{FLX Access}
\end{wrapfigure}

\subsubsection{FLX Access}
\label{subsubsec:FLX Access}

Mit FLX Access stellt Fluxron ein Windows-Tool zum Auslesen und Setzen von gerätespezifischen Parametern bereit. Die Kommunikation erfolgt via Bluetooth. Zudem kann der Gerätestatus ausgelesen werden. 

Dieses System ist besonders für die Entwicklungsphase von Geräten geeignet.

\vspace{2cm}
\WFclear
\begin{wrapfigure}[8]{l}{6cm}
	\includegraphics[scale=0.25]{analysis/res/flxdltool}
	\caption{FLX Downloadtool}
\end{wrapfigure}

\subsubsection{FLX Downloadtool}
\label{subsubsec:FLX Downloadtool}

Das FLX Downloadtool ermöglicht den Download der gesamten Konfiguration in eine Excel-Datei. In dieser können dann die Parameter eingestellt werden. Danach werden die Einstellungen über das Tool wieder auf das Gerät geladen.

Eingesetzt wird dieses Programm für das Schreiben von kundenspezifischen Parametern auf mehrere Geräte.
\WFclear